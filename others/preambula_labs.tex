\documentclass[14pt,final,oneside]{article}% класс документа, характеристики
%>>>>> Разметка документа
\usepackage[a4paper, mag=1000, left=3cm, right=1.5cm, top=2cm, bottom=2cm, headsep=0.7cm, footskip=1cm]{geometry} % По ГОСТу: left>=3cm, right=1cm, top=2cm, bottom=2cm,
\linespread{1} % межстройчный интервал по ГОСТу := 1.5
%<<<<< Разметка документа
\usepackage[utf8]{inputenc}
\usepackage[T2A]{fontenc}
\usepackage[english, russian]{babel}
\usepackage{amsmath, amsfonts, amssymb, amssymb, amsthm, mathtools}
\usepackage{geometry}
\usepackage{colortbl} % таблички
\usepackage{listings} % листинг
\usepackage{dcolumn} % выравнивание чисел
\usepackage[normalem]{ulem} % для подчёркиваний uline
\ULdepth = 0.16em % расстояние от линии до текста выше/ниже
\lstset{basicstyle=\ttfamily\normalsize}  
\usepackage{graphicx}
\usepackage{longtable}
\usepackage[breaklinks]{hyperref}
\usepackage{xcolor}
\usepackage{float}
\makeatletter
\def\cyrrange#1-#2{%
    \begingroup
    \count@=`#1
    \loop
    \lst@literate{\char\count@}{\char\count@}1
    \ifnum\count@<`#2
    \advance\count@\@ne
    \repeat
    \endgroup
}
\makeatother
\lstset{
    inputencoding=utf8,       % Указываем кодировку входного текста
    extendedchars=true,       % Включаем поддержку расширенных символов
    literate=%                % Настройка отображения кириллических символов
        {а}{{\selectfont а}}1
        {б}{{\selectfont б}}1
        {в}{{\selectfont в}}1
        {г}{{\selectfont г}}1
        {д}{{\selectfont д}}1
        {е}{{\selectfont е}}1
        {ё}{{\selectfont ё}}1
        {ж}{{\selectfont ж}}1
        {з}{{\selectfont з}}1
        {и}{{\selectfont и}}1
        {й}{{\selectfont й}}1
        {к}{{\selectfont к}}1
        {л}{{\selectfont л}}1
        {м}{{\selectfont м}}1
        {н}{{\selectfont н}}1
        {о}{{\selectfont о}}1
        {п}{{\selectfont п}}1
        {р}{{\selectfont р}}1
        {с}{{\selectfont с}}1
        {т}{{\selectfont т}}1
        {у}{{\selectfont у}}1
        {ф}{{\selectfont ф}}1
        {х}{{\selectfont х}}1
        {ц}{{\selectfont ц}}1
        {ч}{{\selectfont ч}}1
        {ш}{{\selectfont ш}}1
        {щ}{{\selectfont щ}}1
        {ъ}{{\selectfont ъ}}1
        {ы}{{\selectfont ы}}1
        {ь}{{\selectfont ь}}1
        {э}{{\selectfont э}}1
        {ю}{{\selectfont ю}}1
        {я}{{\selectfont я}}1
        {Б}{{\selectfont Б}}1
        {К}{{\selectfont К}}1
        {О}{{\selectfont О}}1
    language=Python,            % Язык программирования
    numbers=left,             % Нумерация строк слева
    stepnumber=1,             % Каждая строка нумеруется
    numbersep=5pt,            % Отступ от кода до номеров строк
    showspaces=false,         % Не показывать пробелы
    showstringspaces=false,   % Не показывать пробелы в строках
    tabsize=4,                % Размер табуляции
    frame=single,             % Рамка вокруг кода
    breaklines=true,          % Перенос строк
    breakatwhitespace=false,  % Переносить строки по пробелам
    basicstyle=\ttfamily,     % Шрифт текста
    keywordstyle=\color{blue},% Цвет ключевых слов
    commentstyle=\color{green},% Цвет комментариев
    stringstyle=\color{red},  % Цвет строковых литералов
}
\usepackage{minted} 

\usepackage{titlesec}
\titleformat{\section}
{\LARGE\bfseries}
{\thesection}{15pt}{} 
\usepackage{fancyhdr}
\renewcommand{\thesection}{\arabic{section}.}
\renewcommand{\thesubsection}{\arabic{section}.\arabic{subsection}.}
\usepackage{tocloft}
\renewcommand{\cftsecleader}{\cftdotfill{\cftdotsep}}
\renewcommand{\cftsecfont}{\Large\bfseries} 
\renewcommand{\cfttoctitlefont}{\LARGE\bfseries}
\newcommand{\lr}[1]{\left( {#1} \right)}
\usepackage{hyperref} % гиперссылки
\hypersetup{
    colorlinks=true,        % Включаем цветные ссылки
    linkcolor=blue!50!black, % Цвет внутренних ссылок (например, на разделы)
    urlcolor=blue!50!black,  % Цвет URL-ссылок
    citecolor=blue!50!black, % Цвет ссылок на библиографию
}
\usepackage{tikz}

% Команда для обведенного знака равенства
\newcommand{\circledequal}{%
    \mathbin{%
        \tikz[baseline=(X.base)] 
            \node[draw, circle, inner sep=0pt] (X) {$=$};%
    }%
}
