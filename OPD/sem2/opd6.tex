\documentclass[14pt,final,oneside]{article}% класс документа, характеристики
%>>>>> Разметка документа
\usepackage[a4paper, mag=1000, left=3cm, right=1.5cm, top=2cm, bottom=2cm, headsep=0.7cm, footskip=1cm]{geometry} % По ГОСТу: left>=3cm, right=1cm, top=2cm, bottom=2cm,
\linespread{1} % межстройчный интервал по ГОСТу := 1.5
%<<<<< Разметка документа
\usepackage[utf8]{inputenc}
\usepackage[T2A]{fontenc}
\usepackage[english, russian]{babel}
\usepackage{amsmath, amsfonts, amssymb, amssymb, amsthm, mathtools}
\usepackage{geometry}
\usepackage{colortbl} % таблички
\usepackage{listings} % листинг
\usepackage{dcolumn} % выравнивание чисел
\usepackage[normalem]{ulem} % для подчёркиваний uline
\ULdepth = 0.16em % расстояние от линии до текста выше/ниже
\lstset{basicstyle=\ttfamily\normalsize}  
\usepackage{graphicx}
\usepackage{longtable}
\usepackage[breaklinks]{hyperref}
\usepackage{xcolor}
\usepackage{float}
\makeatletter
\def\cyrrange#1-#2{%
    \begingroup
    \count@=`#1
    \loop
    \lst@literate{\char\count@}{\char\count@}1
    \ifnum\count@<`#2
    \advance\count@\@ne
    \repeat
    \endgroup
}
\makeatother
\lstset{
    inputencoding=utf8,       % Указываем кодировку входного текста
    extendedchars=true,       % Включаем поддержку расширенных символов
    literate=%                % Настройка отображения кириллических символов
        {а}{{\selectfont а}}1
        {б}{{\selectfont б}}1
        {в}{{\selectfont в}}1
        {г}{{\selectfont г}}1
        {д}{{\selectfont д}}1
        {е}{{\selectfont е}}1
        {ё}{{\selectfont ё}}1
        {ж}{{\selectfont ж}}1
        {з}{{\selectfont з}}1
        {и}{{\selectfont и}}1
        {й}{{\selectfont й}}1
        {к}{{\selectfont к}}1
        {л}{{\selectfont л}}1
        {м}{{\selectfont м}}1
        {н}{{\selectfont н}}1
        {о}{{\selectfont о}}1
        {п}{{\selectfont п}}1
        {р}{{\selectfont р}}1
        {с}{{\selectfont с}}1
        {т}{{\selectfont т}}1
        {у}{{\selectfont у}}1
        {ф}{{\selectfont ф}}1
        {х}{{\selectfont х}}1
        {ц}{{\selectfont ц}}1
        {ч}{{\selectfont ч}}1
        {ш}{{\selectfont ш}}1
        {щ}{{\selectfont щ}}1
        {ъ}{{\selectfont ъ}}1
        {ы}{{\selectfont ы}}1
        {ь}{{\selectfont ь}}1
        {э}{{\selectfont э}}1
        {ю}{{\selectfont ю}}1
        {я}{{\selectfont я}}1
        {Б}{{\selectfont Б}}1
        {К}{{\selectfont К}}1
        {О}{{\selectfont О}}1
    language=Python,            % Язык программирования
    numbers=left,             % Нумерация строк слева
    stepnumber=1,             % Каждая строка нумеруется
    numbersep=5pt,            % Отступ от кода до номеров строк
    showspaces=false,         % Не показывать пробелы
    showstringspaces=false,   % Не показывать пробелы в строках
    tabsize=4,                % Размер табуляции
    frame=single,             % Рамка вокруг кода
    breaklines=true,          % Перенос строк
    breakatwhitespace=false,  % Переносить строки по пробелам
    basicstyle=\ttfamily,     % Шрифт текста
    keywordstyle=\color{blue},% Цвет ключевых слов
    commentstyle=\color{green},% Цвет комментариев
    stringstyle=\color{red},  % Цвет строковых литералов
}
\usepackage{minted} 

\usepackage{titlesec}
\titleformat{\section}
{\LARGE\bfseries}
{\thesection}{15pt}{} 
\usepackage{fancyhdr}
\renewcommand{\thesection}{\arabic{section}.}
\renewcommand{\thesubsection}{\arabic{section}.\arabic{subsection}.}
\usepackage{tocloft}
\renewcommand{\cftsecleader}{\cftdotfill{\cftdotsep}}
\renewcommand{\cftsecfont}{\Large\bfseries} 
\renewcommand{\cfttoctitlefont}{\LARGE\bfseries}
\newcommand{\lr}[1]{\left( {#1} \right)}
\usepackage{hyperref} % гиперссылки
\hypersetup{
    colorlinks=true,        % Включаем цветные ссылки
    linkcolor=blue!50!black, % Цвет внутренних ссылок (например, на разделы)
    urlcolor=blue!50!black,  % Цвет URL-ссылок
    citecolor=blue!50!black, % Цвет ссылок на библиографию
}
\usepackage{tikz}

% Команда для обведенного знака равенства
\newcommand{\circledequal}{%
    \mathbin{%
        \tikz[baseline=(X.base)] 
            \node[draw, circle, inner sep=0pt] (X) {$=$};%
    }%
}

\begin{document}
\thispagestyle{empty}
\begin{center}
\LARGE{Университет ИТМО} 
\vspace{20pt}

\LARGE{Факультет программной инженерии и компьютерной техники \\
Образовательная программа системное и прикладное программное обеспечение}
\vspace{160pt}

\LARGE{Лабораторная работа  \textnumero 6 \\
По дисциплине "Основы профессиональной деятельности" \\ 
Вариант 9602}
\vspace{120pt}
\end{center}

\begin{flushright}
\LARGE{Выполнил студент группы P3109 \\ 
Евграфов Артём Андреевич \\
Проверила: \\
Ткешелашвили Нино Мерабиевна}
\vspace{120pt}
\end{flushright}

\begin{center}
\Large{Санкт-Петербург 2025}
\end{center}

\newpage
\setcounter{page}{1}
\tableofcontents
\newpage
\section{Задание варианта 9602}
\begin{figure}[H]
    \centering
\includegraphics[width=1\linewidth]{P10.png}
\end{figure}
\section{Реализация задания на ассемблере БЭВМ}
\begin{minted}[linenos=true, frame=single, breaklines=true]{asm}
ORG 0x000
V0: WORD $DEFAULT, 0x180
V1: WORD $INT1, 0x180
V2: WORD $INT2, 0x180
V3: WORD $DEFAULT, 0x180
V4: WORD $DEFAULT, 0x180
V5: WORD $DEFAULT, 0x180
V6: WORD $DEFAULT, 0x180
V7: WORD $DEFAULT, 0x180

ORG 0x00F
DEFAULT: IRET 

ORG 0x034
X:   WORD 0x0000
MIN: WORD 0xFFEC ; -20
MAX: WORD 0x0016 ; 22

START:  CLA ; запрещаем прерывания неиспользуемых ВУ
	OUT 0x1
	OUT 0x7	
	OUT 0xB
	OUT 0xE
	OUT 0x12
	OUT 0x16
	OUT 0x1A
	OUT 0x1E
	
	LD #0x9 ; задаём вектора прерываний для ВУ-1 и ВУ-2
	OUT 0x3
	LD #0xA
	OUT 0x5

MAIN:	DI
	LD X
	INC
	INC
	INC
	CALL $CHECK
	ST X
	EI
	JUMP MAIN

CHECK:
    CMP MIN
    BLT RETURN_MIN
    CMP MAX
    BGE RETURN_MIN
    JUMP RETURN  
    RETURN_MIN: LD $MIN 
    RETURN: RET

LDMAX: LD MAX

INT1:	LD X
	NOP ; для отладки
	ASL
	ASL
	ADD X
	ADD X
	SUB #5
	OUT 0x2
	NOP ; для отладки
	IRET

INT2:	IN 0x4
	NOP ; для отдалки (посмотреть что лежало в ВУ-2)
	NEG
	ST X
	NOP
	IRET

\end{minted}
\section{ОП и ОДЗ исходных данных и результата}
\subsection{Область представления}
X, MIN, MAX – 16-разрядные знаковые числа \\
DR КВУ – 8-разрядное знаковое число
\subsection{Область определения}
$-128 \leq 6x- 5 \leq 127$ \\
\(x \in [-20; 22]\), то есть \(x \in [0\text{xFFEC}; 0\text{x}0016]\).
\section{Проверка корректности}
Проверка основной программы: \\
1. Загрузить код программы в БЭВМ. \\
2. Записать в переменную X максимальное по ОДЗ значение (22) \\
3. Запустить программу в потактовом режиме. \\
4. Когда IP станет равным 46, в AC появится загруженное значение. Когда IP станет равным 49, значение АС увеличится на 3. Если результат не в ОДЗ, то при дальнейшей работы программы АС станет равным 0xFFEC. Если переполнения за ОДЗ не произошло, то при дальшейшей работе программы АС снова увеличится на 3. \\

\begin{tabular}{|c|c|c|}
\hline
X & Ожидание & AC \\
\hline
0x0015 & 0x0018 & 0x0018 \\
\hline
0x0022 & 0\text{xFFEC} & 0\text{xFFEC} \\
\hline
\end{tabular} \\

\noindent Проверка прерывания ВУ-1: \\
1. Загрузить текст программы в БЭВМ. \\
2. Заменить все NOP на HLT. \\
3. Запустить программу в режиме РАБОТА (ТЫК "останов"\, ТЫК пуск). \\
4. Установить «Готовность ВУ-1». \\
5. Дождаться останова. \\
6. Записать значение аккумулятора (значение X) \\
7. Рассчитать ожидаемое значение после обработки прерывания по функции 6x-5. \\
8. Нажать «Продолжение». \\
9. Дождаться останова. \\
10. Записать результат обработки прерывания, регистра DR КВУ-1, и сравнить его с ожидаемым. \\
11. Нажать «Продолжение». \\

\begin{tabular}{|c|c|c|}
\hline
X & Ожидание $6x-5$ & DR ВУ \\
\hline
0x000C & 0x43 & 0x43 \\
\hline
0x0015 & 0x79 & 0x79 \\
\hline
\end{tabular} \\


\noindent ВУ-2: \\ 
1. Ввести в ВУ-2 произвольное число, записать его \\
2. Установить «Готовность ВУ-2». \\
3. Дождаться останова. \\
4. Записать значение аккумулятора. \\
5. Сравнить его с тем числом, которое было введено в ВУ-2 (они должны быть равны). \\
6. Нажать «Продолжение». \\
7. Дождаться останова. \\
8. Ввести в клавишный регистр адрес 0x034 и нажать «ввод адреса», затем «чтение». \\
9. Записать значение DR и сравнить его с тем числом, которое было введено в ВУ-2. Сумма значения DR и числа из ВУ-2 должна давать 0. \\

\begin{tabular}{|c|c|c|}
\hline
DR ВУ & Ожидание $-\text{DR}$ & X \\
\hline
0x0010 & 0xF0 & 0xF0 \\
\hline
0x000С & 0xF4 & 0xF4 \\
\hline
\end{tabular} \\
\section{Вывод}
Во время выполнения данной работы я научился работать с прерываниями на ВУ-1 и ВУ-2. Научился инициализировать векторы прерывания.

\end{document}