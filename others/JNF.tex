\input{preambula_questions}


\title{\centering JNF}  
\author{\centering Евграфов Артём}  
\date{\centering 3 Мая 2025} 

\begin{document}
\maketitle
\newpage
\setcounter{page}{1}
\tableofcontents
\newpage
\section{Условие}
Вариант 17 \\

\noindent \(\displaystyle
\begin{pmatrix}
-7 &   0 &  4 &  1 &  3 &  1 \\
 0 &  -7 & -2 &  0 & -2 & -1 \\
 0 &   0 & -7 &  0 &  1 &  2 \\
 0 &   0 &  0 & -7 &  0 &  0 \\
 0 &   0 &  0 &  0 & -7 &  0 \\
 0 &   0 &  0 &  0 &  0 & -7
\end{pmatrix}
\)
\section{Собственные и присоединенные вектора}
Так как матрица верхнетреугольная, то её характеристический многочлен имеет следующий вид: \(\det(A-\lambda E_6) = (-7 - \lambda)^6.\) У этого многочлена единственный корень $\lambda = -7$ кратности 6. Рассмотрим теперь матрицу $B = A - \lambda E_6$ и уравнение $BX = 0$: \\

\noindent \(\displaystyle
\begin{pmatrix}
0 &   0 &  4 &  1 &  3 &  1 & \vline & 0\\
 0 &  0 & -2 &  0 & -2 & -1 & \vline & 0 \\
 0 &   0 & 0 &  0 &  1 &  2 & \vline & 0\\
 0 &   0 &  0 & 0 &  0 &  0 & \vline & 0\\
 0 &   0 &  0 &  0 & 0 &  0 & \vline & 0\\
 0 &   0 &  0 &  0 &  0 & 0 & \vline & 0
\end{pmatrix}
\implies 
\begin{cases}
x_1 = \beta \\
x_2 = \gamma \\
x_3 = 3\alpha \\
x_4 = -2\alpha \\
x_5 = -4\alpha \\
x_6 = 2\alpha
\end{cases}
\) \\

\noindent Тогда базис $W_1$ состоит из следующих векторов: \\
\[
\begin{pmatrix}
0 \\ 0 \\ 3 \\ -2 \\ -4 \\ 2
\end{pmatrix}
\quad
\begin{pmatrix}
1 \\ 0 \\ 0 \\ 0 \\ 0 \\ 0
\end{pmatrix}
\quad
\begin{pmatrix}
0 \\ 1 \\ 0 \\ 0 \\ 0 \\ 0
\end{pmatrix}
\] \\
Геометрическая кратность собственного значения равна 3,
значит для построения жорданова базиса требуется еще три присоединённых вектора. Найдём их, решив уравнение $B^2X = 0$: \\
\noindent \(\displaystyle
\begin{pmatrix}
0 &   0 &  0 &  0 &  4 &  8 & \vline & 0\\
 0 &  0 & 0 &  0 & -2 & -4 & \vline & 0 \\
 0 &   0 & 0 &  0 &  0 &  0 & \vline & 0\\
 0 &   0 &  0 & 0 &  0 &  0 & \vline & 0\\
 0 &   0 &  0 &  0 & 0 &  0 & \vline & 0\\
 0 &   0 &  0 &  0 &  0 & 0 & \vline & 0
\end{pmatrix}
\implies 
\begin{cases}
x_1 = \beta \\
x_2 = \gamma \\
x_3 = \theta \\
x_4 = \phi \\
x_5 = -2\alpha \\
x_6 =\alpha
\end{cases}
\) \\

\noindent Теперь дополним базис $W_1$ до базиса $W_2$:
\[
\begin{pmatrix}
0 \\ 0 \\ 0 \\ 0 \\ -2 \\ 1
\end{pmatrix}
\quad
\begin{pmatrix}
0 \\ 0 \\ 1 \\ 0 \\ 0 \\ 0
\end{pmatrix}
\quad
\cup W_1
\] \\

\noindent Заметим, что $B^3 = 0$. Тогда в прошлой системе положим $x_5 = 1$ и определим базис $W_3$: \\
\[
\begin{pmatrix}
0 \\ 0 \\ 0 \\ 0 \\ 1 \\ 0
\end{pmatrix}
\quad
\cup W_2
\] \\
\section{Жорданова лестница}
Высота лестницы - 3, $r_3 = 6 - 5 = 1, \; r_2 = 5 - 3 = 2, \; r_{1} = 3$. Вид у жордановой лестницы будет вот такой: \\
\[
\begin{array}{ccccc}
 f & \vline &  & \vline &  \\
 Bf & \vline & g & \vline &  \\
 B^2f & \vline & Bg & \vline & e \\
\end{array}
\] \\
\noindent Верхнюю ступеньку займет $f = (0 \; 0 \; 0 \; 0 \; 1 \;0)^T$, \\

\[
Bf = 
\begin{pmatrix}
0 & 0 & 4 & 1 & 3 & 1 \\
0 & 0 & -2 & 0 & -2 & -1 \\
0 & 0 & 0 & 0 & 1 & 2 \\
0 & 0 & 0 & 0 & 0 & 0 \\
0 & 0 & 0 & 0 & 0 & 0 \\
0 & 0 & 0 & 0 & 0 & 0
\end{pmatrix}
\cdot
\begin{pmatrix}
0 \\
0 \\
0 \\
0 \\
1 \\
0
\end{pmatrix}
=
\begin{pmatrix}
3 \\
-2 \\
1 \\
0 \\
0 \\
0 
\end{pmatrix}
\]
\[
B^2f = 
\begin{pmatrix}
0 & 0 & 0 & 0 & 4 & 8 \\
0 & 0 & 0 & 0 & -2 & -4 \\
0 & 0 & 0 & 0 & 0 & 0 \\
0 & 0 & 0 & 0 & 0 & 0 \\
0 & 0 & 0 & 0 & 0 & 0 \\
0 & 0 & 0 & 0 & 0 & 0
\end{pmatrix}
\cdot
\begin{pmatrix}
0 \\
0 \\
0 \\
0 \\
1 \\
0
\end{pmatrix}
=
\begin{pmatrix}
4 \\
-2 \\
0 \\
0 \\
0 \\
0 
\end{pmatrix}
\]
\noindent На второй ступени положим $g = (0 \; 0 \; 0 \; 0 \; -2 \;1)^T$,
\[
Bg = 
\begin{pmatrix}
0 & 0 & 4 & 1 & 3 & 1 \\
0 & 0 & -2 & 0 & -2 & -1 \\
0 & 0 & 0 & 0 & 1 & 2 \\
0 & 0 & 0 & 0 & 0 & 0 \\
0 & 0 & 0 & 0 & 0 & 0 \\
0 & 0 & 0 & 0 & 0 & 0
\end{pmatrix}
\cdot
\begin{pmatrix}
0 \\
0 \\
0 \\
0 \\
-2 \\
1
\end{pmatrix}
=
\begin{pmatrix}
-5 \\
3 \\
0 \\
0 \\
0 \\
0 
\end{pmatrix}
\]
\noindent На нижнюю ступень положим вектор \(e = (0 \; 0 \; 3 \; -2 \; -4 \;2)^T\). Имеем следующий базис: \\
\[
\begin{pmatrix}
4 \\ -2 \\ 0 \\ 0 \\ 0 \\ 0
\end{pmatrix}
\quad
\begin{pmatrix}
3 \\ -2 \\ 1 \\ 0 \\ 0 \\ 0
\end{pmatrix}
\quad
\begin{pmatrix}
0 \\ 0 \\ 0 \\ 0 \\ 1 \\ 0
\end{pmatrix}
\quad
\begin{pmatrix}
-5 \\ 3 \\ 0 \\ 0 \\ 0 \\ 0
\end{pmatrix}
\quad
\begin{pmatrix}
0 \\ 0 \\ 0 \\ 0 \\ -2 \\ 1
\end{pmatrix}
\quad
\begin{pmatrix}
0 \\ 0 \\ 3 \\ -2 \\ -4 \\ 2
\end{pmatrix}
\] \\
\noindent Имеем матрицу перехода T: \\
\[
\begin{pmatrix}
4 & 3 & 0 & -5 & 0 & 0 \\
-2 & -2 & 0 & 3 & 0 & 0 \\
0 & 1 & 0 & 0 & 0 & 3 \\
0 & 0 & 0 & 0 & 0 & -2 \\
0 & 0 & 1 & 0 & -2 & -4 \\
0 & 0 & 0 & 0 & 1 & 2
\end{pmatrix}
\]
\noindent Определитель этой матрицы равен -4, то есть вектора действительно ЛНЗ. Так как в лестнице 1 столбец высоты 3, один высоты 2 и один высоты 1, то имеем следующую ЖНФ: \\
 \[
     J = \begin{pmatrix}
     -7 & 1 & 0 & 0 & 0 & 0 \\
     0 & -7 & 1 & 0 & 0 & 0 \\
     0 & 0 & -7 & 0 & 0 & 0 \\
     0 & 0 & 0 & -7 & 1 & 0 \\
     0 & 0 & 0 & 0 & -7 & 0 \\
     0 & 0 & 0 & 0 & 0 & -7
     \end{pmatrix}.
     \]
\end{document}