\input{preambula}
\begin{document}
\thispagestyle{empty}
\begin{center}
\LARGE{Университет ИТМО} 
\vspace{20pt}

\LARGE{Факультет программной инженерии и компьютерной техники \\
Образовательная программа системное и прикладное программное обеспечение}
\vspace{160pt}

\LARGE{Лабораторная работа  \textnumero 2 \\
По дисциплине "Информатика" \\ 
Вариант 46\textcolor{red}{5}8\textcolor{red}{2}6} = 52
\vspace{120pt}
\end{center}

\begin{flushright}
\LARGE{Выполнил студент группы P3109 \\
Евграфов Артём Андреевич \\
Проверил:\\
Рыбаков Степан Дмитриевич}
\vspace{120pt}
\end{flushright}

\begin{center}
\Large{Санкт-Петербург 2024}
\end{center}

\newpage
\setcounter{page}{1}
\tableofcontents
\newpage

\section{Задание}
\large
1. Определить свой вариант задания с помощью номера в ISU (он же номер студенческого билета). Вариантом является комбинация 3-й и 5-й цифр. Т.е. если номер в ISU = 12\textcolor{red}{3}4\textcolor{red}{5}6, то вариант = 35. \\
2. На основании номера варианта задания выбрать набор из 4 полученных сообщений в виде последовательности 7-символьного кода. \\
3. Построить схему декодирования классического кода Хэмминга (7;4), которую представить в отчёте в виде изображения. \\
4. Показать, исходя из выбранных вариантов сообщений (по 4 у каждого – часть №1 в варианте), имеются ли в принятом сообщении ошибки, и если имеются, то какие. \textbf{Подробно прокомментировать} и записать правильное сообщение. \\
5. На основании номера варианта задания выбрать 1 полученное сообщение в виде последовательности 11-символьного кода. \\
6. Построить схему декодирования классического кода Хэмминга (15;11), которую представить в отчёте в виде изображения. \\
7. Показать, исходя из выбранного варианта сообщений (по 1 у каждого – часть №2 в варианте), имеются ли в принятом сообщении ошибки, и если имеются, то какие. \textbf{Подробно прокомментировать} и записать правильное сообщение. \\
8. Сложить номера всех 5 вариантов заданий. \uline{\textbf{Умножить полученное число на 4}}. Принять данное число как число информационных разрядов в передаваемом сообщении. Вычислить для данного числа минимальное число проверочных разрядов и коэффициент избыточности. \\
9. \uline{Дополнительное задание №1} (позволяет набрать от 86 до 100 процентов от максимального числа баллов БаРС за данную лабораторную). Написать программу на любом языке программирования, которая на вход получает набор из 7 цифр «0» и «1», записанных подряд, анализирует это сообщение
на основе классического кода Хэмминга (7,4), а затем выдает правильное сообщение (только информационные биты) и 
указывает бит с ошибкой при его наличии.

\section{Основные этапы вычисления}
\subsection{Задание 1 - \textnumero 37}
\begin{flushleft}
    
\begin{tabular}{|c|c|c|c|c|c|c|}
\hline
\cellcolor{orange!40}{$r_1$} & \cellcolor{orange!40}{$r_2$} & $i_1$ & \cellcolor{orange!40}{$r_3$} & $i_2$ & $i_3$ & $i_4$ \\ \hline
\cellcolor{orange!40}{0} & \cellcolor{orange!40}{0} & 0 & \cellcolor{orange!40}{0} & 1 & 1 & 0 \\ \hline 
\end{tabular} \\
\vspace{0.25cm}
$s_1 = r_1 \oplus i_1 \oplus i_2 \oplus i_4 = 0 \oplus 0 \oplus 1 \oplus 0 = 1$ \\
$s_2 = r_2 \oplus i_1 \oplus i_3 \oplus i_4 = 0 \oplus 0 \oplus 1 \oplus 0 = 1$ \\
$s_3 = r_3 \oplus i_2 \oplus i_3 \oplus i_4 = 0 \oplus 1 \oplus 1 \oplus 0 = 0$ \\
\vspace{0.25cm}
\begin{tabular}{|c|c|c|c|c|c|c|c|c|} % создаем колонки со цветом
\hline % горизонтальная линия 
 & 1 & 2 & 3 & 4 & 5 & 6 & 7 &  \\ \hline
$2^x$ & $r_1$ & $r_2$ & $\cellcolor{red!20}i_1$ & $r_3$ & $i_2$ & $i_3$ & $i_4$ & S \\ \hline % заполняем элементы
1 & \cellcolor{blue!20}X & - & \cellcolor{red!20}X & - & \cellcolor{blue!20}X & - & \cellcolor{blue!20}X & $s_1$ \\ \hline % цвет синий, 20% прозрачность
2 & - & \cellcolor{orange!20}X & \cellcolor{red!20}X & - & - & \cellcolor{orange!20}X & - & $s_2$ \\ \hline
4 & - & - & \cellcolor{red!20}- & \cellcolor{green!20}X & \cellcolor{green!20}X & \cellcolor{green!20}X & \cellcolor{green!20}X & $s_3$ \\ \hline
\end{tabular} \\
\vspace{0.25cm} 
$s = \lr{s_1, s_2, s_3} = 100 \Rightarrow \text{ошибка в символе } i_1$ \\
Правильное сообщение: 1110.

\newpage
\subsection{Задание 2 - \textnumero 69}
\begin{tabular}{|c|c|c|c|c|c|c|}
\hline
\cellcolor{orange!40}{$r_1$} & \cellcolor{orange!40}{$r_2$} & $i_1$ & \cellcolor{orange!40}{$r_3$} & $i_2$ & $i_3$ & $i_4$ \\ \hline
\cellcolor{orange!40}{1} & \cellcolor{orange!40}{1} & 1 & \cellcolor{orange!40}{0} & 1 & 0 & 0 \\ \hline 
\end{tabular} \\
\vspace{0.25cm}
$s_1 = r_1 \oplus i_1 \oplus i_2 \oplus i_4 = 1 \oplus 1 \oplus 1 \oplus 0 = 1$ \\
$s_2 = r_2 \oplus i_1 \oplus i_3 \oplus i_4 = 1 \oplus 1 \oplus 0 \oplus 0 = 0$ \\
$s_3 = r_3 \oplus i_2 \oplus i_3 \oplus i_4 = 0 \oplus 1 \oplus 0 \oplus 0 = 1$ \\
\vspace{0.25cm}
\begin{tabular}{|c|c|c|c|c|c|c|c|c|} 
\hline 
 & 1 & 2 & 3 & 4 & 5 & 6 & 7 &  \\ \hline
$2^x$ & $r_1$ & $r_2$ & $i_1$ & $r_3$ & \cellcolor{red!20}$i_2$ & $i_3$ & $i_4$ & S \\ \hline
1 & \cellcolor{blue!20}X & - & \cellcolor{blue!20}X & - & \cellcolor{red!20}X & - & \cellcolor{blue!20}X & $s_1$ \\ \hline
2 & - & \cellcolor{orange!20}X & \cellcolor{orange!20}X & - & \cellcolor{red!20}- & \cellcolor{orange!20}X & - & $s_2$ \\ \hline
4 & - & - & - & \cellcolor{green!20}X & \cellcolor{red!20}X & \cellcolor{green!20}X & \cellcolor{green!20}X & $s_3$ \\ \hline
\end{tabular} \\
\vspace{0.25cm} 
$s = \lr{s_1, s_2, s_3} = 101 \Rightarrow \text{ошибка в символе } i_2$ \\
Правильное сообщение: 1000.

\subsection{Задание 3 - \textnumero 101}
\begin{tabular}{|c|c|c|c|c|c|c|}
\hline
\cellcolor{orange!40}{$r_1$} & \cellcolor{orange!40}{$r_2$} & $i_1$ & \cellcolor{orange!40}{$r_3$} & $i_2$ & $i_3$ & $i_4$ \\ \hline
\cellcolor{orange!40}{0} & \cellcolor{orange!40}{0} & 1 & \cellcolor{orange!40}{1} & 1 & 1 & 1 \\ \hline 
\end{tabular} \\
\vspace{0.25cm}
$s_1 = r_1 \oplus i_1 \oplus i_2 \oplus i_4 = 0 \oplus 1 \oplus 1 \oplus 1 = 1$ \\
$s_2 = r_2 \oplus i_1 \oplus i_3 \oplus i_4 = 0 \oplus 1 \oplus 1 \oplus 1 = 1$ \\
$s_3 = r_3 \oplus i_2 \oplus i_3 \oplus i_4 = 1 \oplus 1 \oplus 1 \oplus 1 = 0$ \\
\vspace{0.25cm}
\begin{tabular}{|c|c|c|c|c|c|c|c|c|} 
\hline 
 & 1 & 2 & 3 & 4 & 5 & 6 & 7 &  \\ \hline
$2^x$ & $r_1$ & $r_2$ & \cellcolor{red!20}$i_1$ & $r_3$ & $i_2$ & $i_3$ & $i_4$ & S \\ \hline
1 & \cellcolor{blue!20}X & - & \cellcolor{red!20}X & - & \cellcolor{blue!20}X & - & \cellcolor{blue!20}X & $s_1$ \\ \hline
2 & - & \cellcolor{orange!20}X & \cellcolor{red!20}X & - & \cellcolor{orange!20}- & \cellcolor{orange!20}X & - & $s_2$ \\ \hline
4 & - & - & \cellcolor{red!20}- & \cellcolor{green!20}X & \cellcolor{green!20}X & \cellcolor{green!20}X & \cellcolor{green!20}X & $s_3$ \\ \hline
\end{tabular} \\
\vspace{0.25cm} 
$s = \lr{s_1, s_2, s_3} = 110 \Rightarrow \text{ошибка в символе } i_1$ \\
Правильное сообщение: 0111.

\subsection{Задание 4 - \textnumero 21}
\begin{tabular}{|c|c|c|c|c|c|c|}
\hline
\cellcolor{orange!40}{$r_1$} & \cellcolor{orange!40}{$r_2$} & $i_1$ & \cellcolor{orange!40}{$r_3$} & $i_2$ & $i_3$ & $i_4$ \\ \hline
\cellcolor{orange!40}{0} & \cellcolor{orange!40}{1} & 1 & \cellcolor{orange!40}{1} & 0 & 0 & 1 \\ \hline 
\end{tabular} \\
\vspace{0.25cm}
$s_1 = r_1 \oplus i_1 \oplus i_2 \oplus i_4 = 0 \oplus 1 \oplus 0 \oplus 1 = 0$ \\
$s_2 = r_2 \oplus i_1 \oplus i_3 \oplus i_4 = 1 \oplus 1 \oplus 0 \oplus 1 = 1$ \\
$s_3 = r_3 \oplus i_2 \oplus i_3 \oplus i_4 = 1 \oplus 0 \oplus 0 \oplus 1 = 0$ \\
\vspace{0.25cm}
\begin{tabular}{|c|c|c|c|c|c|c|c|c|} % 
\hline 
 & 1 & 2 & 3 & 4 & 5 & 6 & 7 &  \\ \hline
$2^x$ & $r_1$ & \cellcolor{red!20}$r_2$ & $i_1$ & $r_3$ & $i_2$ & $i_3$ & $i_4$ & S \\ \hline
1 & \cellcolor{blue!20}X & \cellcolor{red!20}- & \cellcolor{blue!20}X & - & \cellcolor{blue!20}X & - & \cellcolor{blue!20}X & $s_1$ \\ \hline
2 & - & \cellcolor{red!20}X & \cellcolor{orange!20}X & - & \cellcolor{orange!20}- & \cellcolor{orange!20}X & - & $s_2$ \\ \hline
4 & - & \cellcolor{red!20}- & - & \cellcolor{green!20}X & \cellcolor{green!20}X & \cellcolor{green!20}X & \cellcolor{green!20}X & $s_3$ \\ \hline
\end{tabular} \\
\vspace{0.25cm} 
$s = \lr{s_1, s_2, s_3} = 010 \Rightarrow \text{ошибка в символе } r_2$ \\
Правильное сообщение: 1001.

\newpage
\subsection{Задание 5 - \textnumero 52}
\begin{tabular}{|c|c|c|c|c|c|c|c|c|c|c|c|c|c|c|}
\hline
\cellcolor{orange!40}{$r_1$} & \cellcolor{orange!40}{$r_2$} & $i_1$ & \cellcolor{orange!40}{$r_3$} & $i_2$ & $i_3$ & $i_4$ & \cellcolor{orange!40}{$r_4$} & $i_5$ & $i_6$ & $i_7$ & $i_8$ & $i_9$ & $i_{10}$ & $i_{11}$  \\ \hline
\cellcolor{orange!40}{0} & \cellcolor{orange!40}{1} & 0 & \cellcolor{orange!40}{0} & 0 & 1 & 1 & \cellcolor{orange!40}{0} & 1 & 0 & 0 & 0 & 0 & 1 & 1\\ \hline 
\end{tabular} \\
\vspace{0.25cm}
$s_1 = r_1 \oplus i_1 \oplus i_2 \oplus i_4 \oplus i_5 \oplus i_7 \oplus i_9 \oplus i_{11} = 0 \oplus 0 \oplus 0 \oplus 1 \oplus 1 \oplus 0 \oplus 0 \oplus 1 = 1$ \\
$s_2 = r_2 \oplus i_1 \oplus i_3 \oplus i_4 \oplus i_6 \oplus i_7 \oplus i_{10} \oplus i_{11} = 1 \oplus 0 \oplus 1 \oplus 1 \oplus 0 \oplus 0 \oplus 1 \oplus 1 = 1$ \\
$s_3 = r_3 \oplus i_2 \oplus i_3 \oplus i_4 \oplus i_8 \oplus i_9 \oplus i_{10} \oplus i_{11} = 0 \oplus 0 \oplus 1 \oplus 1 \oplus 0 \oplus 0 \oplus 1 \oplus 1 = 0$\\
$s_4 = r_4 \oplus i_5 \oplus i_6 \oplus i_7 \oplus i_8 \oplus i_9 \oplus i_{10} \oplus i_{11} = 0 \oplus 1 \oplus 0 \oplus 0 \oplus 0 \oplus 0 \oplus 1 \oplus 1 = 1$ \\
\vspace{0.25cm}
\begin{tabular}{|c|c|c|c|c|c|c|c|c|c|c|c|c|c|c|c|c|} \hline
 & 1 & 2 & 3 & 4 & 5 & 6 & 7 & 8 & 9 & 10 & 11 & 12 & 13 & 14 & 15 &  \\ \hline
$2^x$ & $r_1$ & $r_2$ & $i_1$ & $r_3$ & $i_2$ & $i_3$ & $i_4$ & $r_4$ & $i_5$ & $i_6$ & \cellcolor{red!20}$i_7$ & $i_8$ & $i_9$ & $i_{10}$ & $i_{11}$ & S \\ \hline
1 & \cellcolor{blue!20}X & - & \cellcolor{blue!20}X & - & \cellcolor{blue!20}X & - & \cellcolor{blue!20}X & - & \cellcolor{blue!20}X & - & \cellcolor{red!20}X & - & \cellcolor{blue!20}X & - & \cellcolor{blue!20}X & $s_1$ \\ \hline
2 & - & \cellcolor{yellow!20}X & \cellcolor{yellow!20}X & - & - & \cellcolor{yellow!20}X & \cellcolor{yellow!20}X & - & - & \cellcolor{yellow!20}X & \cellcolor{red!20}X & - & - & \cellcolor{yellow!20}X & \cellcolor{yellow!20}X & $s_2$ \\ \hline
4 & - & - & - & \cellcolor{green!20}X & \cellcolor{green!20}X & \cellcolor{green!20}X & \cellcolor{green!20}X & - & - & - & \cellcolor{red!20}- & \cellcolor{green!20}X & \cellcolor{green!20}X & \cellcolor{green!20}X & \cellcolor{green!20}X & $s_3$ \\ \hline
8 & - & - & - & - & - & - & - & \cellcolor{orange!20}X & \cellcolor{orange!20}X & \cellcolor{orange!20}X & \cellcolor{red!20}X & \cellcolor{orange!20}X & \cellcolor{orange!20}X & \cellcolor{orange!20}X & \cellcolor{orange!20}X & $s_4$ \\ \hline
\end{tabular} \\
\vspace{0.25cm} 
$s = \lr{s_1, s_2, s_3, s_4} = 1101 \Rightarrow \text{ошибка в символе } i_7$ \\
Правильное сообщение: 00111010011.

\subsection{Задание 6 - \textnumero 1120}
$\lr{37 + 69 + 101 + 21 + 52} \cdot 4 = 1120$ - число информационных разрядов (битов) в принимаемом сообщении. Вычислим минимальное число r контрольных разрядов по формуле 
$2^r \geq r + i + 1, \text{ где i - число информационных битов}$. \\
$2^r \geq r + 1121$. Заметим, что минимальное r, удовлетворяющее условию - \textbf{11}. Вычислим коэффициент избыточности k: \\
$k = \frac{r}{r + i} = \frac{11}{11 + 1120} = \frac{11}{1131} \approx \textbf{0.00972590627}$.

\subsection{Задание 7}
\normalsize
\begin{lstlisting}[caption={Исходный код программы}, language=Java, captionpos=b]
import java.util.Scanner;

public class Hamming {
    private final byte[] chars = new byte[7];
    private final String[] symbols = {"r1", "r2", "i1", 
    "r3", "i2", "i3", "i4"};
    private final byte s1;
    private final byte s2;
    private final byte s3;
    private final String index;
    private final int SysIndex;

    Hamming(String nums) {
        for (int i = 0; i <= 6; i++) {
            chars[i] = (byte) (nums.charAt(i) - '0');
        }
        s1 = (byte) (chars[0] ^ chars[2] ^ chars[4] ^ chars[6]);
        s2 = (byte) (chars[1] ^ chars[2] ^ chars[5] ^ chars[6]);
        s3 = (byte) (chars[3] ^ chars[4] ^ chars[5] ^ chars[6]);
        index = Byte.toString(s1) + s2 + s3;
        SysIndex = (index.charAt(0) - '0') + 
        (index.charAt(1) - '0') * 2 + (index.charAt(2) - '0') * 4;
    }

    void PrintIndex() {
        if (SysIndex == 0) {
            System.out.println("No errors");
        } else {
            System.out.println("Wrong bit is " + symbols[SysIndex - 1]);
        }
    }

    void PrintCorrectly() {
        if (SysIndex == 0) {
            System.out.println("The message is correct");
        } else {
            chars[SysIndex - 1] = (byte) (1 - chars[SysIndex - 1]);
            System.out.println("Correct message is: " 
            chars[2] + chars[4] + chars[5] + chars[6]);
        }
    }

    public static void main(String[] args) {
        Scanner console = new Scanner(System.in);
        String name = console.nextLine();
        Hamming m = new Hamming(name);
        m.PrintCorrectly();
        m.PrintIndex();
    }
}
\end{lstlisting}

\section{Вывод}
В процессе выполнения лабораторной работы я разобрался в работе кода Хэмминга, улучшил навыки программирования на новом для меня язык Java, научился создавать таблицы на языке вёрстки \LaTeX.

\section{Список использованных источников}
1. \textbf{AGalilov (название YouTube канала)}, Код Хэмминга. Самоконтролирующийся и самокорректирующийся код. – URL: \href{https://youtu.be/QsBYshN5idw?si=iddXwSZEYuyY0KgW}{https://youtu.be/QsBYshN5idw?si=iddXwSZEYuyY0KgW}. \\
2. \textbf{П.В. Балакшин, В.В. Соснин, И.В. Калинин, Т.А. Малышева, С.В. Раков, Н.Г. Рущенко, А.М. Дергачев} Информатика: лабораторные работы и тесты [Электронный ресурс]. - URL: \href{https://t.me/balakshin_students}{https://t.me/balakshin\_students}.
\end{flushleft}
\end{document}
