\documentclass[14pt,final,oneside]{article}% класс документа, характеристики
%>>>>> Разметка документа
\usepackage[a4paper, mag=1000, left=3cm, right=1.5cm, top=2cm, bottom=2cm, headsep=0.7cm, footskip=1cm]{geometry} % По ГОСТу: left>=3cm, right=1cm, top=2cm, bottom=2cm,
\linespread{1} % межстройчный интервал по ГОСТу := 1.5
%<<<<< Разметка документа
\usepackage[utf8]{inputenc}
\usepackage[T2A]{fontenc}
\usepackage[english, russian]{babel}
\usepackage{amsmath, amsfonts, amssymb, amssymb, amsthm, mathtools}
\usepackage{geometry}
\usepackage{colortbl} % таблички
\usepackage{listings} % листинг
\usepackage{dcolumn} % выравнивание чисел
\usepackage[normalem]{ulem} % для подчёркиваний uline
\ULdepth = 0.16em % расстояние от линии до текста выше/ниже
\lstset{basicstyle=\ttfamily\normalsize}  
\usepackage{graphicx}
\usepackage{longtable}
\usepackage[breaklinks]{hyperref}
\usepackage{xcolor}
\usepackage{float}
\makeatletter
\def\cyrrange#1-#2{%
    \begingroup
    \count@=`#1
    \loop
    \lst@literate{\char\count@}{\char\count@}1
    \ifnum\count@<`#2
    \advance\count@\@ne
    \repeat
    \endgroup
}
\makeatother
\lstset{
    inputencoding=utf8,       % Указываем кодировку входного текста
    extendedchars=true,       % Включаем поддержку расширенных символов
    literate=%                % Настройка отображения кириллических символов
        {а}{{\selectfont а}}1
        {б}{{\selectfont б}}1
        {в}{{\selectfont в}}1
        {г}{{\selectfont г}}1
        {д}{{\selectfont д}}1
        {е}{{\selectfont е}}1
        {ё}{{\selectfont ё}}1
        {ж}{{\selectfont ж}}1
        {з}{{\selectfont з}}1
        {и}{{\selectfont и}}1
        {й}{{\selectfont й}}1
        {к}{{\selectfont к}}1
        {л}{{\selectfont л}}1
        {м}{{\selectfont м}}1
        {н}{{\selectfont н}}1
        {о}{{\selectfont о}}1
        {п}{{\selectfont п}}1
        {р}{{\selectfont р}}1
        {с}{{\selectfont с}}1
        {т}{{\selectfont т}}1
        {у}{{\selectfont у}}1
        {ф}{{\selectfont ф}}1
        {х}{{\selectfont х}}1
        {ц}{{\selectfont ц}}1
        {ч}{{\selectfont ч}}1
        {ш}{{\selectfont ш}}1
        {щ}{{\selectfont щ}}1
        {ъ}{{\selectfont ъ}}1
        {ы}{{\selectfont ы}}1
        {ь}{{\selectfont ь}}1
        {э}{{\selectfont э}}1
        {ю}{{\selectfont ю}}1
        {я}{{\selectfont я}}1
        {Б}{{\selectfont Б}}1
        {К}{{\selectfont К}}1
        {О}{{\selectfont О}}1
    language=Python,            % Язык программирования
    numbers=left,             % Нумерация строк слева
    stepnumber=1,             % Каждая строка нумеруется
    numbersep=5pt,            % Отступ от кода до номеров строк
    showspaces=false,         % Не показывать пробелы
    showstringspaces=false,   % Не показывать пробелы в строках
    tabsize=4,                % Размер табуляции
    frame=single,             % Рамка вокруг кода
    breaklines=true,          % Перенос строк
    breakatwhitespace=false,  % Переносить строки по пробелам
    basicstyle=\ttfamily,     % Шрифт текста
    keywordstyle=\color{blue},% Цвет ключевых слов
    commentstyle=\color{green},% Цвет комментариев
    stringstyle=\color{red},  % Цвет строковых литералов
}
\usepackage{minted} 

\usepackage{titlesec}
\titleformat{\section}
{\LARGE\bfseries}
{\thesection}{15pt}{} 
\usepackage{fancyhdr}
\renewcommand{\thesection}{\arabic{section}.}
\renewcommand{\thesubsection}{\arabic{section}.\arabic{subsection}.}
\usepackage{tocloft}
\renewcommand{\cftsecleader}{\cftdotfill{\cftdotsep}}
\renewcommand{\cftsecfont}{\Large\bfseries} 
\renewcommand{\cfttoctitlefont}{\LARGE\bfseries}
\newcommand{\lr}[1]{\left( {#1} \right)}
\usepackage{hyperref} % гиперссылки
\hypersetup{
    colorlinks=true,        % Включаем цветные ссылки
    linkcolor=blue!50!black, % Цвет внутренних ссылок (например, на разделы)
    urlcolor=blue!50!black,  % Цвет URL-ссылок
    citecolor=blue!50!black, % Цвет ссылок на библиографию
}
\usepackage{tikz}

% Команда для обведенного знака равенства
\newcommand{\circledequal}{%
    \mathbin{%
        \tikz[baseline=(X.base)] 
            \node[draw, circle, inner sep=0pt] (X) {$=$};%
    }%
}

\begin{document}
\thispagestyle{empty}
\begin{center}
\LARGE{Университет ИТМО} 
\vspace{20pt}

\LARGE{Факультет программной инженерии и компьютерной техники \\
Образовательная программа системное и прикладное программное обеспечение}
\vspace{160pt}

\LARGE{Лабораторная работа  \textnumero 7 \\
По дисциплине "Основы профессиональной деятельности" \\ 
Вариант 9702}
\vspace{120pt}
\end{center}

\begin{flushright}
\LARGE{Выполнил студент группы P3109 \\ 
Евграфов Артём Андреевич \\
Проверила: \\
Ткешелашвили Нино Мерабиевна}
\vspace{120pt}
\end{flushright}

\begin{center}
\Large{Санкт-Петербург 2025}
\end{center}

\newpage
\setcounter{page}{1}
\tableofcontents
\newpage
\section{Задание варианта 9702}
\begin{figure}[H]
    \centering
\includegraphics[width=1\linewidth]{P1O.png}
\end{figure}
\section{Микрокомандный код}
\begin{longtable}{|>{\centering\arraybackslash}p{1cm}|>{\centering\arraybackslash}p{2.5cm}|>{\centering\arraybackslash}p{3cm}|>{\arraybackslash}p{9cm}|}
\hline
Адрес МП & Микрокоманда & Описание & Комментарий \\
\hline
\endfirsthead

\hline
\endlastfoot
E0 & 0010809501 & $\sim$ DR + 1 → AC, N, Z & Вся команда  \\\hline
E1 & 80C4101040 & GOTO INT @ C4 & Переход на цикл прерывания \\\hline
\end{longtable}
\section{Трассировка микрокоманды}
\begin{longtable}{|*{11}{>{\ttfamily}r|}}
\hline
\textbf{МР до выборки МК} & \textbf{МR} & \textbf{IP} & \textbf{CR} & \textbf{AR} & \textbf{DR} & \textbf{SP} & \textbf{BR} & \textbf{AC} & \textbf{NZVC} & \textbf{Мр (СчМК)} \\ 
\hline
\endhead
 &  & 020 & 0000 & 000 & 0000 & 000 & 0000 & 0000 & 0000 &  \\\hline
 &  & 021 & 0000 & 020 & 0010 & 000 & 0000 & 0000 & 0000 &  \\\hline
 &  & 022 & 0000 & 021 & 9020 & 000 & 0000 & 0000 & 0000 &  \\\hline
 &  & 021 & 0000 & 021 & 9020 & 000 & 0000 & 0000 & 0000 &  \\\hline
D4 & 00BBE00000 & 021 & 0000 & 000 & 0000 & 000 & 0000 & 0000 & 0100 & D5 \\\hline
D5 & 80C3101040 & 021 & 0000 & 000 & 0000 & 000 & 0000 & 0000 & 0100 & C3 \\\hline
C3 & 0400000000 & 021 & 0000 & 000 & 0000 & 000 & 0000 & 0000 & 0100 & C4 \\\hline
C4 & 80DE801040 & 021 & 0000 & 000 & 0000 & 000 & 0000 & 0000 & 0100 & DE  \\\hline
DE & 4000000000 & 021 & 0000 & 000 & 0000 & 000 & 0000 & 0000 & 0100 & DF  \\\hline
DF & 8001101040 & 021 & 0000 & 000 & 0000 & 000 & 0000 & 0000 & 0100 & 01  \\\hline
01 & 00A0009004 & 021 & 0000 & 021 & 0000 & 000 & 0021 & 0000 & 0100 & 02  \\\hline
02 & 0104009420 & 022 & 0000 & 021 & 9020 & 000 & 0021 & 0000 & 0100 & 03  \\\hline
03 & 0002009001 & 022 & 9020 & 021 & 9020 & 000 & 0021 & 0000 & 0100 & 04  \\\hline
04 & 8109804002 & 022 & 9020 & 021 & 9020 & 000 & 0021 & 0000 & 0100 & 09  \\\hline
09 & 800C404002 & 022 & 9020 & 021 & 9020 & 000 & 0021 & 0000 & 0100 & 0C  \\\hline
0C & 8024084002 & 022 & 9020 & 021 & 9020 & 000 & 0021 & 0000 & 0100 & 24  \\\hline
24 & 8026804002 & 022 & 9020 & 021 & 9020 & 000 & 0021 & 0000 & 0100 & 25  \\\hline
25 & 814A404002 & 022 & 9020 & 021 & 9020 & 000 & 0021 & 0000 & 0100 & 26  \\\hline
26 & 0080009001 & 022 & 9020 & 020 & 9020 & 000 & 0021 & 0000 & 0100 & 27  \\\hline
27 & 0100000000 & 022 & 9020 & 020 & 0010 & 000 & 0021 & 0000 & 0100 & 28  \\\hline
28 & 813C804002 & 022 & 9020 & 020 & 0010 & 000 & 0021 & 0000 & 0100 & 3C  \\\hline
3C & 8143204002 & 022 & 9020 & 020 & 0010 & 000 & 0021 & 0000 & 0100 & 3D  \\\hline
3D & 81E0104002 & 022 & 9020 & 020 & 0010 & 000 & 0021 & 0000 & 0100 & E0  \\\hline
E0 & 0010809501 & 022 & 9020 & 020 & 0010 & 000 & 0021 & FFF0 & 1000 & E1  \\\hline
\end{longtable}
\newpage
\section{Тестовая программа}
\begin{minted}[linenos=true, frame=single, breaklines=true]{asm}
ORG 0x0444
RES1: WORD 0x0000
RES2: WORD 0x0000
RES3: WORD 0x0000
X: WORD 0x0000
Y: WORD 0x00FF

TEST1:
	LD #0xB
	ST $X
	LD $X
	AND $Y
	ST $X
	WORD 0x9447
	WORD 0xF201
	JUMP $TEST2
	NEG
	SUB $X
	BZS SET1
	JUMP $TEST2

SET1:
	LD #1
	ST $RES1
	JUMP $TEST2

TEST2:
	LD #0xFA
	ST $X
	LD $X
	AND $Y
	ST $X
	WORD 0x9447
	WORD 0xF201
	JUMP $TEST3
	NEG
	SUB $X
	BZS SET2
	JUMP $TEST3
	
SET2:
	LD #0x1
	ST $RES2
	JUMP $TEST3

TEST3:
	LD #0x80
	ST $X
	LD $X
	AND $Y
	ST $X
	WORD 0x9447
	WORD 0xF201
	JUMP $TEST4
	NEG
	SUB $X
	BZS SET3
	JUMP $TEST4

SET3:
	LD #0x1
	ST $RES3
	JUMP $TEST4

TEST4:
	ST $RES1
	AND $RES2
	AND $RES3
	HLT
\end{minted}
\section{Методика проверки команды}
1. Запустить БЭВМ с помощью команды java -jar -Dmode=cli bcomp-ng.jar \\ 
2. Записать мои микрокоманды в память с помощью команд ma и mw: \\
ma E0 \\
mw 0010809501 \\
mw 80C4101040 \\
mdecodea \\
3. Записать тестовую программу в память: прописать команду asm, потом вставить тестовую
программу и в конце написать END. \\
4. Прописать start C C C C C ... ($\geq 40$ букв C). \\
5. Если в AC единица, значит все тесты пройдены. \\
\section{Вывод}
В ходе выполнения данной лабораторной работы я познакомился с МПУ БЭВМ и синтезировал собственную команду.
\end{document}