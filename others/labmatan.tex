\documentclass[14pt,final,oneside]{article}% класс документа, характеристики
%>>>>> Разметка документа
\usepackage[a4paper, mag=1000, left=3cm, right=1.5cm, top=2cm, bottom=2cm, headsep=0.7cm, footskip=1cm]{geometry} % По ГОСТу: left>=3cm, right=1cm, top=2cm, bottom=2cm,
\linespread{1} % межстройчный интервал по ГОСТу := 1.5
%<<<<< Разметка документа
\usepackage[utf8]{inputenc}
\usepackage[T2A]{fontenc}
\usepackage[english, russian]{babel}
\usepackage{amsmath, amsfonts, amssymb, amssymb, amsthm, mathtools}
\usepackage{geometry}
\usepackage{colortbl} % таблички
\usepackage{listings} % листинг
\usepackage{dcolumn} % выравнивание чисел
\usepackage[normalem]{ulem} % для подчёркиваний uline
\ULdepth = 0.16em % расстояние от линии до текста выше/ниже
\lstset{basicstyle=\ttfamily\normalsize}  
\usepackage{graphicx}
\usepackage{xcolor}
\usepackage{float}
\lstset{
    language=Java,
    numbers=left,                    % Нумерация строк слева
    stepnumber=1,                    % Каждая строка нумеруется
    numbersep=5pt,                   % Отступ от кода до номеров строк
    showspaces=false,                % Не показывать пробелы
    showstringspaces=false,          % Не показывать пробелы в строках
    tabsize=4,                       % Размер табуляции
    frame=single
    breaklines=true,                 % Перенос строк
    breakatwhitespace=false,         % Переносить строки по пробелам
    basicstyle=\ttfamily,            % Шрифт текста
    keywordstyle=\color{blue},       % Цвет ключевых слов
    commentstyle=\color{green},      % Цвет комментариев
    stringstyle=\color{red},         % Цвет строковых литералов
}
\usepackage{titlesec}
\titleformat{\section}
{\LARGE\bfseries}
{\thesection}{15pt}{} 
\usepackage{fancyhdr}
\theoremstyle{theorem}
\newtheorem{theorem}{Теорема}
\newtheorem{lemma}{Лемма}
\renewcommand{\thesection}{\arabic{section}.}
\renewcommand{\thesubsection}{\arabic{section}.\arabic{subsection}.}
\usepackage{tocloft}
\renewcommand{\cftsecleader}{\cftdotfill{\cftdotsep}}
\renewcommand{\cftsecfont}{\Large\bfseries} 
\renewcommand{\cfttoctitlefont}{\LARGE\bfseries}
\newcommand{\lr}[1]{\left( {#1} \right)}
\usepackage{hyperref} % гиперссылки
\hypersetup{
    colorlinks=true,        % Включаем цветные ссылки
    linkcolor=blue!50!black, % Цвет внутренних ссылок (например, на разделы)
    urlcolor=blue!50!black,  % Цвет URL-ссылок
    citecolor=blue!50!black, % Цвет ссылок на библиографию
}


\begin{document}
\thispagestyle{empty}
\begin{center}
\LARGE{Университет ИТМО} 
\vspace{20pt}

\vspace{180pt}

\LARGE{Численное интегрирование \\
Евграфов Артём, 465826, P3109\\ 
Вариант 18 \\}
\vspace{335pt}
\end{center}

\begin{center}
\Large{Санкт-Петербург 2025}
\end{center}

\newpage
\setcounter{page}{1}
\noindent Вычислить приближённо интеграл с погрешностью $\epsilon = 0,00001$  методами: прямоугольников, трапеций, Симпсона. 
\[\int_{\frac{\pi}{4}}^{\frac{\pi}{2}} \ln(\sin{x}) \; dx\] \\

\noindent \textbf{Метод прямоугольников} \\
\(f'(x) = \cot{x}\) - производная непрерывна на отрезке, можно воспользоваться оценкой погрешности: \(\displaystyle |\Delta| \leq \frac{(b-a)}{4} h \sup_{x \in[\frac{\pi}{4}, \frac{\pi}{2}]} |f'(x)|.\) \\
\(\displaystyle \sup_{x \in [\frac{\pi}{2},\frac{\pi}{4}]} |f'(x)| = f'(\frac{\pi}{4}) = 1\), так как $f'(x)$ - убывающая функция на отрезке. \\
$|\Delta| \leq \frac{\pi}{16}h, \; \frac{16|\Delta|}{\pi} \leq h$, значит, чтобы получить погрешность меньше 0,00001, нужно выбрать такой h
$h \leq \frac{16 \cdot 0.00001}{\pi} \approx 0.000050929$.
Пусть $h = 0.00005, $ так как $h = \frac{b - a}{n}$, то $n \approx 15708$. \\
Пусть $x_i = \left(i + \frac{1}{2} \right)h$, где $0 \leq i \leq n - 1$. \\
\[\int_{\frac{\pi}{4}}^{\frac{\pi}{2}} \ln(\sin{x}) \; dx \approx h\sum_{i = 0}^{n - 1}\ln{\left(\sin{\left(x_i\right)}\right)} \approx -0.08641372538311086\]
\begin{minted}[linenos=true, frame=single, breaklines=true]{python}
from math import pi, log, sin

a = pi / 4
b = pi / 2
h = 0.00005
n = int((b - a) / h)
s = 0

for i in range(n):
    xi = a + (i + 0.5) * h
    s += log(sin(xi))

result = s * h
print(result)
\end{minted}

\noindent \textbf{Метод трапеций} \\
Пусть $x_i = ih$, где $0 \leq i \leq n$, тогда $\displaystyle \int_a^{b}f(x) \, dx \approx h\left(\frac{1}{2}\left(f(x_0)  + f(x_n)\right) +\sum_{i = 0}^{n - 1}f(x_i) \right)$ - формула трапеций. \\
\(\displaystyle |\Delta| \leq \frac{(b-a)}{4} h \sup_{x \in[a,b]} |f'(x)|\). Аналогично методу прямоугольников, пусть $h = 0,00005$ и $n = 15708$. \\
\[\displaystyle \int_{\frac{\pi}{4}}^{\frac{\pi}{2}}\ln{\left( \sin{\left( x \right)}\right)} \approx h\left(\frac{1}{2}\left(\ln{\left( \sin{\left( \frac{\pi}{2} \right)}\right)} + \ln{\left( \sin{\left( \frac{\pi}{4}\right)}\right)} \right) + \sum_{i = 0}^{n-1}{\ln{\left( \sin{\left( x_i \right)}\right)}}\right) \approx -0.08641372569556682\]
\begin{minted}[linenos=true, frame=single, breaklines=true]{python}
from math import pi, log, sin

a = pi / 4
b = pi / 2
h = 0.00005
n = int((b - a) / h)
s = 0.5 * (log(sin(a)) + log(sin(b)))

for i in range(1, n):
    xi = a + i * h
    s += log(sin(xi))

result = s * h
print(result)
\end{minted}
\textbf{Метод Симпсона}
\[
\int_{a}^{b} f(x) \, dx \approx \frac{h}{3} \left( f(x_0) + f(x_n) + 2 \sum_{i=1}^{m-1} f(x_{2i}) + 4 \sum_{i=1}^{m} f(x_{2i-1}) \right),
\]
При $n = 2m$. \\
\(\displaystyle |\Delta| \leq \frac{5}{18}(b-a)h\sup_{x \in[a, b]} |f'(x)|\), \\
\(|\Delta| \leq \frac{5\pi h}{72}\), \\
\(\frac{72|\Delta|}{5\pi} \leq h\).
Для минимизации погрешности выберем $h \leq \frac{72\cdot0.00001}{5\pi} \approx 0.0000458366.$ \\
Пусть опять $h = 0.00005, \; h = \frac{b - a}{n} \approx 15708, \; m = 7854.$ \\
Пусть $x_i = ih,$ где $0 \leq i \leq n$. \\
\(\displaystyle \int_{\frac{\pi}{4}}^{\frac{\pi}{2}} \ln(\left(\sin{\left( x\right)} \right)) \; dx \approx \frac{h}{3}\left(\ln(\left(\sin{\left( \frac{\pi}{2}\right)} 
\right)) + \ln(\left(\sin{\left( \frac{\pi}{4}\right)} \right))  + 2\sum_{i=1}^{m-1}\ln(\left(\sin{\left( {x_{2i}}\right)} \right))+ 4\sum_{i=1}^{m}\ln(\left(\sin{\left( {x_{2i-1}}\right)} \right))\right) \approx -0.0864137254870533.\)
\begin{minted}[linenos=true, frame=single, breaklines=true]{python}
from math import pi, log, sin

a = pi / 4
b = pi / 2
h = 0.00005
n = int((b - a) / h)  
s = log(sin(a)) + log(sin(b))  

for i in range(1, n // 2):
    xi = a + 2 * i * h
    s += 2 * log(sin(xi))

for i in range(1, n // 2 + 1):
    xi = a + (2 * i - 1) * h
    s += 4 * log(sin(xi))

result = h / 3 * s
print(result)
\end{minted}
\textbf{Вывод}: точным значением интеграла является -0.08641372548729102... \\
Наиболее точным оказался метод Симпсона (погрешность $2.3772651\cdot10^{-13}$).
\end{document}