\documentclass[14pt,final,oneside]{article}% класс документа, характеристики
%>>>>> Разметка документа
\usepackage[a4paper, mag=1000, left=1cm, right=1cm, top=1cm, bottom=1cm, headsep=0.7cm, footskip=1cm]{geometry} % По ГОСТу: left>=3cm, right=1cm, top=2cm, bottom=2cm,
\linespread{1} % межстройчный интервал по ГОСТу := 1.5
%<<<<< Разметка документа
\usepackage[utf8]{inputenc}
\usepackage[T2A]{fontenc}
\usepackage[english, russian]{babel}
\usepackage{amsmath, amsfonts, amssymb, amssymb, amsthm, mathtools}
\usepackage{geometry}
\pagenumbering{gobble} % откл нумерацию
\usepackage{colortbl} % таблички
\usepackage{listings} % листинг
\usepackage{dcolumn} % выравнивание чисел
\usepackage[normalem]{ulem} % для подчёркиваний uline
\ULdepth = 0.16em % расстояние от линии до текста выше/ниже
\lstset{basicstyle=\ttfamily\normalsize}  
\usepackage{graphicx}
\usepackage{xcolor}
\usepackage{float}
\lstset{
    language=Java,
    numbers=left,                    % Нумерация строк слева
    stepnumber=1,                    % Каждая строка нумеруется
    numbersep=5pt,                   % Отступ от кода до номеров строк
    showspaces=false,                % Не показывать пробелы
    showstringspaces=false,          % Не показывать пробелы в строках
    tabsize=4,                       % Размер табуляции
    breaklines=true,                 % Перенос строк
    breakatwhitespace=false,         % Переносить строки по пробелам
    basicstyle=\ttfamily,            % Шрифт текста
    keywordstyle=\color{blue},       % Цвет ключевых слов
    commentstyle=\color{green},      % Цвет комментариев
    stringstyle=\color{red},         % Цвет строковых литералов
}
\usepackage{titlesec}
\titleformat{\section}
{\LARGE\bfseries}
{\thesection}{15pt}{} 
\usepackage{fancyhdr}
\theoremstyle{theorem}
\newtheorem{theorem}{Теорема}
\newtheorem{lemma}{Лемма}
\renewcommand{\thesection}{\arabic{section}.}
\renewcommand{\thesubsection}{\arabic{section}.\arabic{subsection}.}
\usepackage{tocloft}
\renewcommand{\cftsecleader}{\cftdotfill{\cftdotsep}}
\renewcommand{\cftsecfont}{\Large\bfseries} 
\renewcommand{\cfttoctitlefont}{\LARGE\bfseries}
\newcommand{\lr}[1]{\left( {#1} \right)}
\usepackage{hyperref} % гиперссылки
\hypersetup{
    colorlinks=true,        % Включаем цветные ссылки
    linkcolor=blue!50!black, % Цвет внутренних ссылок (например, на разделы)
    urlcolor=blue!50!black,  % Цвет URL-ссылок
    citecolor=blue!50!black, % Цвет ссылок на библиографию
}


\begin{document}
\noindent SELECT столбцы \\
FROM таблица \\
WHERE условия \\
GROUP BY столбцы\_группировки\\
HAVING условия\_для\_групп \\
ORDER BY сортировка\ \\

\noindent WHERE — фильтрация строк до группировки \\
Назначение: Отбирает строки из таблицы, которые удовлетворяют указанным условиям. \\
Особенности: \\
Работает с отдельными строками, а не с группами. \\
Не может использовать агрегатные функции (SUM, COUNT, AVG), так как они применяются после группировки. \\

\noindent GROUP BY — это оператор SQL, который группирует строки с одинаковыми значениями в указанных столбцах. После группировки можно применять агрегатные функции (например, COUNT, SUM, AVG, MAX, MIN) для анализа данных внутри каждой группы. Можно группировать данные по комбинации столбцов: \\
SELECT department, position, COUNT(*) as count \\
FROM employees \\
GROUP BY department, position; \\

\noindent В SELECT можно указывать только: \\
столбцы из GROUP BY, \\
агрегатные функции (например, SUM, COUNT). \\
Если добавить столбец, не входящий в GROUP BY, возникнет ошибка. \\

\noindent CUBE расширяет обычную группировку, создавая все возможные комбинации группируемых столбцов, включая итоги. NULL в столбцах группировки означает, что итог рассчитан без учета этого столбца. \\

\noindent ROLLUP создает иерархические итоги (например, для иерархии "год → месяц → день"). ROLLUP (a, b, c) → группы: (a,b,c), (a,b), (a), (). NULL в столбце означает, что это итоговая строка для предыдущего уровня иерархии. \\
GROUPING(column) возвращает: \\
0, если столбец участвует в текущей группировке, \\
1, если столбец не участвует (итоговая строка). \\
\begin{figure}[H]
    \centering
    \includegraphics[width=0.4\linewidth]{BD4.png}
\end{figure}

\noindent Ключевое слово DISTINCT в SQL используется для возврата уникальных значений из столбцов в результатах запроса. Оно помогает исключить дубликаты, оставив только неповторяющиеся строки. DISTINCT применяется ко всем столбцам, перечисленным после SELECT. Возвращает только уникальные комбинации значений этих столбцов. \\

\noindent HAVING — фильтрация групп после группировки \\
Назначение: Отбирает группы, которые удовлетворяют условию. \\
Особенности: \\
Работает с агрегированными данными (например, COUNT(*), AVG(salary)). \\
Всегда используется после GROUP BY. \\
SELECT department, AVG(salary) as avg\_salary \\
FROM employees \\
GROUP BY department \\
HAVING AVG(salary) > 60000; \\

\noindent Комбинация GROUP BY и агрегатных функций в SQL позволяет анализировать данные на уровне групп, а не отдельных строк. Агрегатные функции (COUNT, SUM, AVG, MAX, MIN) применяются к каждой группе отдельно. \\
SELECT department, COUNT(*) as employees\_count \\
FROM employees \\
GROUP BY department; \\
\begin{figure}[H]
    \centering
    \includegraphics[width=0.2\linewidth]{BD1.png}
\end{figure}

\noindent Агрегатные функции в SQL — это функции, которые выполняют вычисления над набором значений и возвращают единственное значение. Они используются для анализа данных, например, для подсчета количества записей, вычисления суммы, среднего значения, поиска максимума или минимума. Агрегатные функции нельзя использовать в WHERE. Для фильтрации групп применяйте HAVING. \\
SELECT department, AVG(salary) as avg\_salary, COUNT(*) as employees\_count \\
FROM employees \\
WHERE hire\_date >= '2020-01-01'  -- Сотрудники, принятые после 2020 \\
GROUP BY department              -- Группировка по отделам \\
HAVING AVG(salary) > 60000      -- Отделы с средней зарплатой > 60k \\
ORDER BY avg\_salary DESC;       -- Сортировка от высокой к низкой зарплате \\

\noindent Если что-то сравнивается с null, то результат - null. \\

\noindent -- Найти имена, начинающиеся на "Ан" \\
SELECT * FROM users WHERE name LIKE 'Ан\%'; (LIKE - с учётом регистра, ILIKE - нет) \\

\noindent SIMILAR TO - для сравнения с регулярками POSIX. \\

\noindent Конструкция CASE WHEN THEN в SQL позволяет выполнять условные проверки и возвращать различные значения в зависимости от выполнения условий. Это аналог оператора if-else в других языках программирования.  \\
CASE \\
    WHEN условие1 THEN результат1 \\
    WHEN условие2 THEN результат2 \\
    ... \\
    ELSE результат\_по\_умолчанию \\
END \\


\noindent SELECT name, salary, \\
CASE \\
WHEN salary > 100000 THEN 'Высокая' \\
WHEN salary BETWEEN 50000 AND 100000 THEN 'Средняя' \\
ELSE 'Низкая' \\
END AS salary\_category \\
FROM employees; \\
\begin{figure}[H]
    \centering
    \includegraphics[width=0.2\linewidth]{BD2.png}
\end{figure}

\noindent CAST(выражение AS целевой\_тип) - преобразование типов \\
выражение::целевой\_тип - тоже (чисто для PostgreSQL) \\
SELECT '3.14'::FLOAT; -- Результат: 3.14 \\

\noindent Запросы с несколькими таблицами: \\
1. СУБД формирует строки декартова произведения таблиц, перечисленных во фразе FROM. \\
2. СУБД проверяет, удовлетворяют ли данные в сформированной строке условиям из фразы WHERE. \\
3. Если данные в строке удовлетворяют условию из фразы WHERE, то СУБД включает в ответ на запрос те поля строки, которые соответствуют столбцам, перечисленным во фразе SELECT. \\

\noindent INNER JOIN \\
Описание: \\
Объединяет строки из двух таблиц только при наличии совпадений по условию. Соединения (JOIN) часто более читаемы и лучше оптимизируются СУБД, особенно для эквисоединений. \\
SELECT users.name, orders.amount \\
FROM users \\
INNER JOIN orders ON users.id = orders.user\_id; \\
Результат: \\
Только пользователи с заказами. \\

\noindent LEFT OUTER JOIN (или LEFT JOIN) \\
Описание: \\
Возвращает все строки из левой таблицы и совпадающие строки из правой. Если совпадений нет, в правой части — NULL. \\
SELECT users.name, orders.amount \\
FROM users \\
LEFT JOIN orders ON users.id = orders.user\_id; \\
Результат: \\
Все пользователи, включая тех, у кого нет заказов (в amount будет NULL). Если в таблице groups есть несколько строк с одинаковым id, соответствующим id юзера, то в итоговой выборке каждому юзеру будет сопоставлена каждая из найденных групп. Это приведет к дублированию строк студента для каждого совпадения в users.\\ 

\noindent RIGHT OUTER JOIN (или RIGHT JOIN) \\
Возвращает все строки из правой таблицы и совпадающие строки из левой. Если совпадений нет, в левой части — NULL. \\
SELECT users.name, orders.amount \\
FROM users \\
RIGHT JOIN orders ON users.id = orders.user\_id; \\
Результат: \\
Все заказы, включая те, у которых нет пользователя (в name будет NULL). \\ 

\noindent FULL OUTER JOIN (или FULL JOIN) \\
Возвращает все строки из обеих таблиц. Если совпадений нет, недостающие части заполняются NULL (комбинация LEFT JOIN и RIGHT JOIN). \\
SELECT users.name, orders.amount \\
FROM users \\
FULL JOIN orders ON users.id = orders.user\_id; \\ 
Результат: \\
Все пользователи и все заказы, включая несовпадающие записи. \\

\noindent CROSS JOIN \\
Создает декартово произведение строк из двух таблиц (каждая строка из первой объединяется с каждой строкой из второй). Обычно не является желаемым результатом, так как создает много “бессмысленных” \, комбинаций строк и очень ресурсоёмко. \\

\noindent Вложенный подзапрос (subquery) — предложение SELECT, которое заключено в круглые скобки и вложено в WHERE/HAVING часть другого SQL- предложения. Обрабатываются "снизу вверх": первым обрабатывается вложенный подзапрос самого нижнего уровня.
\begin{figure}[H]
    \centering
    \includegraphics[width=0.5\linewidth]{BD3.png}
\end{figure}
\noindent Аналогично: \\
SELECT Surname \\
FROM STUDENT \\ 
JOIN CITIES ON CityName = City \\
WHERE CITIES.Country = 'Россия' \\

\noindent Сначала выполняется JOIN (объединение таблиц). \\
Затем WHERE (фильтрация строк). \\
Потом GROUP BY (группировка). \\
После этого HAVING (фильтрация групп). \\
И только в конце — SELECT. \\

\noindent а) Некоррелированный подзапрос \\
Не зависит от данных внешнего запроса. \\
Выполняется один раз перед внешним запросом. \\

\noindent б) Зависит от данных внешнего запроса. \\
Выполняется для каждой строки внешнего запроса. \\
SELECT name \\
FROM employees e \\
WHERE salary > (SELECT AVG(salary) \\
FROM employees \\
WHERE department = e.department); \\

\noindent IN (аналогично ANY, SOME) \\
Подзапрос должен возвращать ровно один столбец. \\
Результат выражения сравнивается с каждым
значением, возвращённым подзапросом. \\
Результатом выражения IN будет «true», если значение выражения соответствует хотя бы одному значению, которое вернул подзапрос;  \\
Если выражение: NULL или соответствие в правой части не найдено, и есть хотя бы одно NULL-значение в результате подзапроса, то результат - NULL. \\
\begin{figure}[H]
    \centering
    \includegraphics[width=0.5\linewidth]{BD5.png}
\end{figure}

\noindent EXISTS \\
EXISTS принимает оператор SELECT (подзапрос) \\
Если подзапрос возвращает хотя бы одну строку, то результатом EXISTS будет «true», а если не возвращает ни одной — «false». \\
Если подзапрос возвращает NULL, результат - «true». \\
Подзапрос SELECT 1 — это технический прием, так как важен сам факт наличия строк, а не их содержимое. Если студент имеет несколько экзаменов с оценкой <60, он все равно выведется один раз (дубликаты фамилий не создаются, если они не дублируются в таблице STUDENT). Использование EXISTS эффективнее, чем IN, так как проверка останавливается при нахождении первой подходящей строки.
\begin{figure}[H]
    \centering
    \includegraphics[width=0.5\linewidth]{BD6.png}
\end{figure}

\noindent ALL (эквивалент NOT IN, <> - не равно) \\
Подзапрос должен возвращать ровно один столбец \\
Значение выражения сравнивается со значением в каждой строке результата подзапроса с помощью оператора, который должен возвращать логическое значение. \\
Результатом ALL будет «true», если условие истинно для всех строк (ANY - для какой-то) (и когда подзапрос не возвращает строк), и «false», если находятся строки, для которых оно ложно. \\
Результат будет NULL, если сравнение не возвращает «false» ни для одной из строк, но как минимум для одной результат сравнения при применении оператора NULL. \\
SELECT Name, Salary \\
FROM Employees \\
WHERE Salary > ALL ( \\
    SELECT Salary  \\
    FROM Employees  \\
    WHERE Department = 'IT' \\
); \\

\noindent TRUNCATE() или TRUNC() — отбрасывание знаков: обрезает число до указанного количества знаков после запятой (без округления, FORMAT() — с округлением).

\noindent numeric - числа с заданной точностью \\

\noindent Представления (VIEWs) — это виртуальные таблицы (по сути это именованны подзапрос), которые хранят результат выполнения запроса. Они ведут себя как обычные таблицы, но данные в них не хранятся физически, а формируются динамически при каждом обращении. Вместо написания сложного SQL-запроса каждый раз можно создать представление и обращаться к нему, как к обычной таблице. Можно ограничить доступ к определённым данным (по сути через представления можно давать ограниченный доступ к таблице кому-то, давая её 'копию'), создавая представления, которые показывают только нужные столбцы или строки.
\begin{figure}[H]
    \centering
    \includegraphics[width=0.3\linewidth]{BD7.png}
\end{figure}
\noindent CREATE VIEW active\_users AS \\
SELECT id, name, email FROM users WHERE status = 'active'; \\
SELECT * FROM active\_users;
\begin{figure}[H]
    \centering
    \includegraphics[width=0.3\linewidth]{BD8.png}
\end{figure}

\noindent Материализованные представления хранят результат запроса в виде таблицы и могут быть обновлены вручную: \\
CREATE MATERIALIZED VIEW user\_stats AS \\
SELECT status, COUNT(*) FROM users GROUP BY status; \\
Обновление: \\
REFRESH MATERIALIZED VIEW user\_stats; \\
Материализованное представление быстрее при чтении, но не обновляется автоматически (нужно выполнять REFRESH). 

\noindent Обычные представления нельзя изменять (INSERT, UPDATE, DELETE), но с INSTEAD OF триггерами можно сделать их редактируемыми. \\

\noindent В PostgreSQL представления можно изменять с помощью команды: \\
CREATE OR REPLACE VIEW имя\_представления AS новый\_запрос; \\
Этот метод позволяет обновлять представление без удаления и повторного создания, что удобно при обновлении логики запросов. Новое представление полностью заменит старое. Если на представление ссылаются другие объекты (например, другие представления), то замена может вызвать ошибки. Материализованные представления нельзя заменить через CREATE OR REPLACE. \\

\noindent PL/pgSQL (Procedural Language/PostgreSQL) — это процедурный язык программирования, встроенный в PostgreSQL. Он позволяет писать функции, триггеры и хранимые процедуры, расширяя возможности SQL логикой программирования. Код PL/pgSQL обычно пишется внутри функций или анонимных блоков. Общий шаблон: \\
DECLARE \\
  -- Объявление переменных (опционально) \\
\noindent BEGIN \\
  -- Основной код \\
  -- SQL-запросы, циклы, условия, обработка ошибок \\
\noindent EXCEPTION \\
  -- Обработка ошибок (опционально) \\
\noindent END; \\

\noindent Последовательности (Sequences) в SQL — это объекты базы данных, предназначенные для генерации уникальных числовых значений. Они часто используются для создания автоинкрементных идентификаторов (например, первичных ключей) или других последовательностей чисел. \\
CREATE SEQUENCE sequence\_name \\
    INCREMENT BY increment    -- шаг приращения (по умолчанию 1) \\
    START WITH start          -- начальное значение (по умолчанию 1) \\
    MINVALUE min              -- минимальное значение \\
    MAXVALUE max              -- максимальное значение \\
    CYCLE | NO CYCLE         -- перезапуск при достижении максимума (без цикла - ошибка) \\
    CACHE cache;              -- кэширование значений для повышения производительности \\

\noindent Получить следующее значение: \\ 
SELECT nextval('seq'); \\
Текущее значение: \\
SELECT currval('seq'); \\
Сбросить последовательность: \\
ALTER SEQUENCE user\_id\_seq RESTART WITH 50; \\
Последовательности часто используются для автоинкрементных полей: \\
CREATE TABLE users ( \\
    id INT PRIMARY KEY DEFAULT nextval('user\_id\_seq'), \\
    name TEXT \\
); \\

\noindent smallserial - serial на основе smallint (2 байта)\\
serial - на основе интов (4 байта)\\
bigserial - на основе бигинт (8) \\

\noindent CREATE SEQUENCE user\_id\_seq \\
    START WITH 1000 \\
    INCREMENT BY 1  \\
    MINVALUE 1000   \\
    MAXVALUE 9999  \\
    CYCLE \\
CREATE TABLE users ( \\
    user\_id INT DEFAULT nextval('user\_id\_seq') PRIMARY KEY, \\
    name VARCHAR(50) NOT NULL \\
); \\

\noindent INSERT INTO users (name) VALUES  ('Alice'); -- user\_id = 1000 \\
INSERT INTO users (name) VALUES ('Bob');   -- user\_id = 1001   \\ 

\noindent Функция \\
Возвращает одно значение (число, строку, таблицу и т.д.) на основе входных параметров. \\
Используется в SQL-запросах \\
Процедура \\
Выполняет действия с данными (вставка, обновление, удаление) или бизнес-логику. \\
Не возвращает значение напрямую (может использовать OUT-параметры). Вызывается через CALL. \\
Транзакция \\
Обеспечивает атомарность группы операций: либо все выполняются, либо ни одна. \\
Начинается с BEGIN, завершается COMMIT или ROLLBACK. \\
Гарантирует целостность данных при сбоях. \\

\noindent USING (colum\_list): Упрощенная запись для эквисоединения, когда столбцы, по которым идет соединение, имеют одинаковые имена в обеих таблицах. Столбцы из списка column\_list включаются в результат только один раз. \\

\noindent Предполагая, что в обеих таблицах есть колонка gr\_id \\
FROM students JOIN groups USING (gr\_id) \\
Эквивалентно ON students.gr\_id = groups.gr\_id, но в SELECT будет только одна колонка gr\_id. \\

\noindent NATURAL JOIN: Еще более короткая запись. Автоматически соединяет таблицы по всем столбцам с одинаковыми именами. Совпадающие столбцы включаются в результат только один раз. \\
FROM students NATURAL JOIN groups \\ 

\noindent Ограничение UNIQUE гарантирует, что все значения в столбце (или комбинации столбцов) уникальны. \\
CREATE TABLE orders ( \\
    order\_id INT, \\
    product\_id INT, \\
    UNIQUE (order\_id, product\_id) -- Комбинация order\_id и product\_id должна быть уникальной \\
); \\

\noindent ALTER TABLE users  \\
ADD CONSTRAINT unique\_email UNIQUE (email); \\

\noindent ALTER TABLE users  \\
DROP CONSTRAINT unique\_email; \\ 

\noindent DELETE - удаляет строки из таблицы \\
DROP - полностью удаляет объект \\
TRUNCATE - очищает таблицу
\end{document}