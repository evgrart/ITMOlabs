\documentclass[14pt,final,oneside]{article}% класс документа, характеристики
%>>>>> Разметка документа
\usepackage[a4paper, mag=1000, left=3cm, right=1.5cm, top=2cm, bottom=2cm, headsep=0.7cm, footskip=1cm]{geometry} % По ГОСТу: left>=3cm, right=1cm, top=2cm, bottom=2cm,
\linespread{1} % межстройчный интервал по ГОСТу := 1.5
%<<<<< Разметка документа
\usepackage[utf8]{inputenc}
\usepackage[T2A]{fontenc}
\usepackage[english, russian]{babel}
\usepackage{amsmath, amsfonts, amssymb, amssymb, amsthm, mathtools}
\usepackage{geometry}
\usepackage{colortbl} % таблички
\usepackage{listings} % листинг
\usepackage{dcolumn} % выравнивание чисел
\usepackage[normalem]{ulem} % для подчёркиваний uline
\ULdepth = 0.16em % расстояние от линии до текста выше/ниже
\lstset{basicstyle=\ttfamily\normalsize}  
\usepackage{graphicx}
\usepackage{xcolor}
\usepackage{float}
\lstset{
    language=Java,
    numbers=left,                    % Нумерация строк слева
    stepnumber=1,                    % Каждая строка нумеруется
    numbersep=5pt,                   % Отступ от кода до номеров строк
    showspaces=false,                % Не показывать пробелы
    showstringspaces=false,          % Не показывать пробелы в строках
    tabsize=4,                       % Размер табуляции
    frame=single
    breaklines=true,                 % Перенос строк
    breakatwhitespace=false,         % Переносить строки по пробелам
    basicstyle=\ttfamily,            % Шрифт текста
    keywordstyle=\color{blue},       % Цвет ключевых слов
    commentstyle=\color{green},      % Цвет комментариев
    stringstyle=\color{red},         % Цвет строковых литералов
}
\usepackage{titlesec}
\titleformat{\section}
{\LARGE\bfseries}
{\thesection}{15pt}{} 
\usepackage{fancyhdr}
\theoremstyle{theorem}
\newtheorem{theorem}{Теорема}
\newtheorem{lemma}{Лемма}
\renewcommand{\thesection}{\arabic{section}.}
\renewcommand{\thesubsection}{\arabic{section}.\arabic{subsection}.}
\usepackage{tocloft}
\renewcommand{\cftsecleader}{\cftdotfill{\cftdotsep}}
\renewcommand{\cftsecfont}{\Large\bfseries} 
\renewcommand{\cfttoctitlefont}{\LARGE\bfseries}
\newcommand{\lr}[1]{\left( {#1} \right)}
\usepackage{hyperref} % гиперссылки
\hypersetup{
    colorlinks=true,        % Включаем цветные ссылки
    linkcolor=blue!50!black, % Цвет внутренних ссылок (например, на разделы)
    urlcolor=blue!50!black,  % Цвет URL-ссылок
    citecolor=blue!50!black, % Цвет ссылок на библиографию
}

\begin{document}
\thispagestyle{empty}
\begin{center}
\LARGE{Университет ИТМО} 
\vspace{20pt}

\LARGE{Факультет программной инженерии и компьютерной техники \\
Образовательная программа системное и прикладное программное обеспечение}
\vspace{160pt}

\LARGE{Лабораторная работа  \textnumero 3 \\
По дисциплине "Основы профессиональной деятельности" \\ 
Вариант 9003}
\vspace{120pt}
\end{center}

\begin{flushright}
\LARGE{Выполнил студент группы P3109 \\ 
Евграфов Артём Андреевич \\
Проверила: \\
Ткешелашвили Нино Мерабиевна}
\vspace{120pt}
\end{flushright}

\begin{center}
\Large{Санкт-Петербург 2025}
\end{center}

\newpage
\setcounter{page}{1}
\tableofcontents
\newpage
\section{Задание варианта 9003}
\begin{figure}[H]
    \centering
    \includegraphics[width=0.4\linewidth]{P1O.png}
\end{figure}
\section{Описание программы}
Программа считает количества неотрицательных элементов массива. \\
(i) означает, что команда выполняется до i-ого срабатывания JUMP.
\begin{table}[H]
\centering
\begin{tabular}{|>{\centering\arraybackslash}p{1cm}|>{\centering\arraybackslash}p{3cm}|>{\centering\arraybackslash}p{3cm}|>{\arraybackslash}p{8cm}|}
\hline
Адрес & Содержимое & Мнемоника & Комментарии \\\hline
5C2 & 05D2 & A & Адрес начала массива  \\\hline
5C3 & A000 (05D2, 05D3, 05D4, 05D5) & M & Адрес текущего элемента массива \\\hline
5C4 & 4000 (0003, 0002, 0001, 0000) & L & Количество элементов массива \\\hline
5C5 & E000 (0000) & R & Результат \\\hline
5C6 & +0200 & CLA & Обнуление AC, установка N = 0 \\\hline
5C7 & EEFD & ST (IP - 3)  & (1) Переходим на адрес 5С5 и AC -> 5C5 (AC = 0000) \\\hline
5C8 & AF03 & LD \#0003 & (1) 0003 -> AC \\\hline
5C9 & EEFA & ST (IP - 6) & (1) Переходим на адрес 5С4 и AC -> 5С4 (AC = 0003) \\\hline
5CA & AEF7 & LD (IP - 9) & (1) Переходим на адрес 5С2 и 0x5C2 -> AC (0x5C2 = 05D2) \\\hline
5CB & EEF7 & ST (IP - 9) & (1) Переходим на адрес 5С3 и AC -> 0x5C3 (AC = 05D2) \\\hline
5CC & AAF6 & LD (IP - A)+ & (1) Переходим на адрес 5С3 (F600) и 0x5C3 -> AC; (2) Переходим на адрес 5С3 (FE00) и 0x5C3 -> AC; (3) Переходим на адрес 5С3 (F800) и 0x5C3 -> AC \\\hline
5CD & F201 & BNS (IP + 1)  & (1) Переходим на адрес 5СF; (2) Переходим на адрес 5СF; (3) Переходим на адрес 5СF \\\hline
5CE & 3AF6 & OR (IP - 9)+ & R увеличивается на 1 вследствие того, что адресация автоинкрементная  \\\hline
5CF & 85C4 & LOOP 0x5C4 & (1) Декрементируем 5С4; (2) Декрементируем 5С4; (3) Декрементируем 5С4 \\\hline
5D0 & CEFB & JUMP (IP - 5) & (1) Переходим на адрес 5СС; (2) Переходим на адрес 5СС \\\hline
5D1 & 0100 & HLT & (3) Завершение работы программы \\\hline
5D2 & F600 & $\text{L}_{0}$ &  Элемент массива \\\hline
5D3 & FE00 & $\text{L}_{1}$ & Элемент массива \\\hline
5D4 & F800 & $\text{L}_{2}$ & Элемент массива \\\hline
\end{tabular}
\end{table}

\section{ОП и ОДЗ исходных данных и результата}
\subsection{Область представления}
$\text{L}_{0}$, $\text{L}_{1}$, $\text{L}_{2}$ - 16-разрядные знаковые числа \\
A - 11-разрядное знаковое число \\
М - 11-разрядное знаковое число \\
L - 8-разрядное беззнаковое число \\
R - 16-разрядное беззнаковое число \\

\subsection{Область определения}
\[
\begin{cases}
    0 \leq R \leq 127, \\
    -2^{15} \leq L_i \leq 2^{15} - 1, \\
    0 < L \leq 127, \\
    A \in [0, (5C2)_{16} - L] \cup [(5D2)_{16}, (7FF)_{16}], \\
    A \leq M \leq A + L - 1.
\end{cases}
\]

\section{Трассировка программы}

\begin{center}
\begin{tabular}{|c|c|c|c|c|c|c|c|c|c|c|c|}
\hline
\textbf{Адр} & \textbf{Знч} & \textbf{IP} & \textbf{CR} & \textbf{AR} & \textbf{DR} & \textbf{SP} & \textbf{BR} & \textbf{AC} & \textbf{NZVC} & \textbf{Адр} & \textbf{Знч} \\ \hline
5B0 & 0000 & 5B0 & 0000 & 000 & 0000 & 000 & 0000 & 0000 & 0100 &  &  \\ \hline
5B0 & 0000 & 5B1 & 0000 & 5B0 & 0000 & 000 & 05B0 & 0000 & 0100 &  &  \\ \hline
5B1 & 0047 & 5B2 & 00A7 & 5B1 & 00A7 & 000 & 05B1 & 0000 & 0100 &  &  \\ \hline
5B2 & FFD4 & 5B3 & FFD4 & 5B2 & FFD4 & 000 & 05B2 & 0000 & 0100 &  &  \\ \hline
5B3 & 0000 & 5B4 & 0000 & 5B3 & 0000 & 000 & 05B3 & 0000 & 0100 &  &  \\ \hline
5B4 & 0000 & 5B5 & 0000 & 5B4 & 0000 & 000 & 05B4 & 0000 & 0100 &  &  \\ \hline
5B5 & 0000 & 5B6 & 0000 & 5B6 & 0000 & 000 & 05B5 & 0000 & 0100 &  &  \\ \hline
5B6 & 0000 & 5B7 & 0000 & 5B6 & 0000 & 000 & 05B6 & 0000 & 0100 &  &  \\ \hline
5B7 & 0000 & 5B8 & 0000 & 5B7 & 0000 & 000 & 05B7 & 0000 & 0100 &  &  \\ \hline
5B8 & 0000 & 589 & 0000 & 5B8 & 0000 & 000 & 05BB & 0000 & 0100 &  &  \\ \hline
5B9 & 0000 & 5BA & 0000 & 5B9 & 0000 & 000 & 05B9 & 0000 & 0100 &  &  \\ \hline
5BA & 0000 & 5BB & 0000 & 5BA & 0000 & 000 & 05BA & 0000 & 0100 &  &  \\ \hline
5BB & 0000 & 5BC & 0000 & 5BB & 0000 & 000 & 05BB & 0000 & 0100 &  &  \\ \hline
5BC & 0000 & 5BD & 0000 & 5BC & 0000 & 000 & 05BC & 0000 & 0100 &  &  \\ \hline
5BD & 0000 & 5BE & 0000 & 5BD & 0000 & 000 & 05BD & 0000 & 0100 &  &  \\ \hline
5BE & 0000 & 5BF & 0000 & 5BE & 0000 & 000 & 05BE & 0000 & 0100 &  &  \\ \hline
5BF & 0000 & 5C0 & 0000 & 5BF & 0000 & 000 & 05BF & 0000 & 0100 &  &  \\ \hline
5C0 & 0000 & 5C1 & 0000 & 5C0 & 0000 & 000 & 05C0 & 0000 & 0100 &  &  \\ \hline
5C1 & 0000 & 5C2 & 0000 & 5C1 & 0000 & 000 & 05C1 & 0000 & 0100 &  &  \\ \hline
5C2 & 05B0 & 5C3 & 05B0 & 5C2 & 05B0 & 000 & 05C2 & 0000 & 0100 &  &  \\ \hline
5C3 & A000 & 5C4 & A000 & 000 & 0000 & 000 & 05C3 & 0000 & 0100 &  &  \\ \hline
5C4 & 4000 & 5C5 & 4000 & 000 & 0000 & 000 & 05C4 & 0000 & 0100 &  &  \\ \hline
5C5 & E000 & 5C6 & E000 & 000 & 0000 & 000 & 05C5 & 0000 & 0100 & 000 & 0000 \\ \hline
5C6 & 0200 & 5C7 & 0200 & 5C6 & 0200 & 000 & 05C6 & 0000 & 0100 &  &  \\ \hline
5C7 & EEFD & 5C8 & EEFD & 5C5 & 0000 & 000 & FFFD & 0000 & 0100 & 5C5 & 0000 \\ \hline
5C8 & AF03 & 5C9 & AF03 & 5C8 & 0003 & 000 & 0003 & 0003 & 0000 &  &  \\ \hline
5C9 & EEFA & 5CA & EEFA & 5C4 & 0003 & 000 & FFFA & 0003 & 0000 & 5C4 & 0003 \\ \hline
5CA & AEF7 & 5CB & AEF7 & 5C2 & 05B0 & 000 & FFF7 & 05B0 & 0000 &  &  \\ \hline
5CB & EEF7 & 5CC & EEF7 & 5C3 & 05B0 & 000 & FFF7 & 05B0 & 0000 & 5C3 & 05B0 \\ \hline
5CC & AAF6 & 5CD & AAF6 & 5B0 & 0000 & 000 & FFF6 & 0000 & 0100 & 5C3 & 05B1 \\ \hline
5CD & F201 & 5CE & F201 & 5CD & F201 & 000 & 05CD & 0000 & 0100 &  &  \\ \hline
5CE & 3AF6 & 5CF & 3AF6 & 000 & 0000 & 000 & FFFF & 0000 & 0100 & 5C5 & 0001 \\ \hline
5CF & 85C4 & 5D0 & 85C4 & 5C4 & 0002 & 000 & 0001 & 0000 & 0100 & 5C4 & 0002 \\ \hline
5D0 & CEFB & 5CC & CEFB & 5D0 & 05CC & 000 & FFFB & 0000 & 0100 &  &  \\ \hline
5CC & AAF6 & 5CD & AAF6 & 5B1 & 00A7 & 000 & FFF6 & 00A7 & 0000 & 5C3 & 05B2 \\ \hline
5CD & F201 & 5CE & F201 & 5CD & F201 & 000 & 05CD & 00A7 & 0000 &  &  \\ \hline
5CE & 3AF6 & 5CF & 3AF6 & 001 & 0000 & 000 & FF58 & 00A7 & 0000 & 5C5 & 0002 \\ \hline
5CF & 85C4 & 5D0 & 85C4 & 5C4 & 0001 & 000 & 0000 & 00A7 & 0000 & 5C4 & 0001 \\ \hline
5D0 & CEFB & 5CC & CEFB & 5D0 & 05CC & 000 & FFFB & 00A7 & 0000 &  &  \\ \hline
5CC & AAF6 & 5CD & AAF6 & 5B2 & FFD4 & 000 & FFF6 & FFD4 & 1000 & 5C3 & 0583 \\ \hline
5CD & F201 & 5CF & F201 & 5CD & F201 & 000 & 0001 & FFD4 & 1000 &  &  \\ \hline
5CF & 85C4 & 5D1 & 85C4 & 5C4 & 0000 & 000 & FFFF & FFD4 & 1000 & 5C4 & 0000 \\ \hline
5D1 & 0100 & 5D2 & 0100 & 5D1 & 0100 & 000 & 05D1 & FFD4 & 1000 &  &  \\ \hline
5D2 & 0000 & 5D3 & 0000 & 5D2 & 0000 & 000 & 05D2 & FFD4 & 1000 &  &  \\ \hline
5D3 & 0000 & 5D4 & 0000 & 5D3 & 0000 & 000 & 05D3 & FFD4 & 1000 &  &  \\ \hline
5D4 & 0000 & 5D5 & 0000 & 5D4 & 0000 & 000 & 05D4 & FFD4 & 1000 &  &  \\ \hline
\end{tabular}
\end{center}

\section{Вывод}
В ходе выполнения лабораторной работы я научился работать с командами ветвления, циклами, массивами, также повторил режимы адресации.


\end{document}