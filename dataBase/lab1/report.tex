\documentclass[14pt,final,oneside]{article}% класс документа, характеристики
%>>>>> Разметка документа
\usepackage[a4paper, mag=1000, left=3cm, right=1.5cm, top=2cm, bottom=2cm, headsep=0.7cm, footskip=1cm]{geometry} % По ГОСТу: left>=3cm, right=1cm, top=2cm, bottom=2cm,
\linespread{1} % межстройчный интервал по ГОСТу := 1.5
%<<<<< Разметка документа
\usepackage[utf8]{inputenc}
\usepackage[T2A]{fontenc}
\usepackage[english, russian]{babel}
\usepackage{amsmath, amsfonts, amssymb, amssymb, amsthm, mathtools}
\usepackage{geometry}
\usepackage{colortbl} % таблички
\usepackage{listings} % листинг
\usepackage{dcolumn} % выравнивание чисел
\usepackage[normalem]{ulem} % для подчёркиваний uline
\ULdepth = 0.16em % расстояние от линии до текста выше/ниже
\lstset{basicstyle=\ttfamily\normalsize}  
\usepackage{graphicx}
\usepackage{longtable}
\usepackage[breaklinks]{hyperref}
\usepackage{xcolor}
\usepackage{float}
\makeatletter
\def\cyrrange#1-#2{%
    \begingroup
    \count@=`#1
    \loop
    \lst@literate{\char\count@}{\char\count@}1
    \ifnum\count@<`#2
    \advance\count@\@ne
    \repeat
    \endgroup
}
\makeatother
\lstset{
    inputencoding=utf8,       % Указываем кодировку входного текста
    extendedchars=true,       % Включаем поддержку расширенных символов
    literate=%                % Настройка отображения кириллических символов
        {а}{{\selectfont а}}1
        {б}{{\selectfont б}}1
        {в}{{\selectfont в}}1
        {г}{{\selectfont г}}1
        {д}{{\selectfont д}}1
        {е}{{\selectfont е}}1
        {ё}{{\selectfont ё}}1
        {ж}{{\selectfont ж}}1
        {з}{{\selectfont з}}1
        {и}{{\selectfont и}}1
        {й}{{\selectfont й}}1
        {к}{{\selectfont к}}1
        {л}{{\selectfont л}}1
        {м}{{\selectfont м}}1
        {н}{{\selectfont н}}1
        {о}{{\selectfont о}}1
        {п}{{\selectfont п}}1
        {р}{{\selectfont р}}1
        {с}{{\selectfont с}}1
        {т}{{\selectfont т}}1
        {у}{{\selectfont у}}1
        {ф}{{\selectfont ф}}1
        {х}{{\selectfont х}}1
        {ц}{{\selectfont ц}}1
        {ч}{{\selectfont ч}}1
        {ш}{{\selectfont ш}}1
        {щ}{{\selectfont щ}}1
        {ъ}{{\selectfont ъ}}1
        {ы}{{\selectfont ы}}1
        {ь}{{\selectfont ь}}1
        {э}{{\selectfont э}}1
        {ю}{{\selectfont ю}}1
        {я}{{\selectfont я}}1
        {Б}{{\selectfont Б}}1
        {К}{{\selectfont К}}1
        {О}{{\selectfont О}}1
    language=Python,            % Язык программирования
    numbers=left,             % Нумерация строк слева
    stepnumber=1,             % Каждая строка нумеруется
    numbersep=5pt,            % Отступ от кода до номеров строк
    showspaces=false,         % Не показывать пробелы
    showstringspaces=false,   % Не показывать пробелы в строках
    tabsize=4,                % Размер табуляции
    frame=single,             % Рамка вокруг кода
    breaklines=true,          % Перенос строк
    breakatwhitespace=false,  % Переносить строки по пробелам
    basicstyle=\ttfamily,     % Шрифт текста
    keywordstyle=\color{blue},% Цвет ключевых слов
    commentstyle=\color{green},% Цвет комментариев
    stringstyle=\color{red},  % Цвет строковых литералов
}
\usepackage{minted} 

\usepackage{titlesec}
\titleformat{\section}
{\LARGE\bfseries}
{\thesection}{15pt}{} 
\usepackage{fancyhdr}
\renewcommand{\thesection}{\arabic{section}.}
\renewcommand{\thesubsection}{\arabic{section}.\arabic{subsection}.}
\usepackage{tocloft}
\renewcommand{\cftsecleader}{\cftdotfill{\cftdotsep}}
\renewcommand{\cftsecfont}{\Large\bfseries} 
\renewcommand{\cfttoctitlefont}{\LARGE\bfseries}
\newcommand{\lr}[1]{\left( {#1} \right)}
\usepackage{hyperref} % гиперссылки
\hypersetup{
    colorlinks=true,        % Включаем цветные ссылки
    linkcolor=blue!50!black, % Цвет внутренних ссылок (например, на разделы)
    urlcolor=blue!50!black,  % Цвет URL-ссылок
    citecolor=blue!50!black, % Цвет ссылок на библиографию
}
\usepackage{tikz}

% Команда для обведенного знака равенства
\newcommand{\circledequal}{%
    \mathbin{%
        \tikz[baseline=(X.base)] 
            \node[draw, circle, inner sep=0pt] (X) {$=$};%
    }%
}

\begin{document}
\thispagestyle{empty}
\begin{center}
\LARGE{Университет ИТМО} 
\vspace{20pt}

\LARGE{Факультет программной инженерии и компьютерной техники \\
Образовательная программа системное и прикладное программное обеспечение}
\vspace{160pt}

\LARGE{Лабораторная работа  \textnumero 1 \\
По дисциплине "Базы данных" \\ 
Вариант 1986}
\vspace{120pt}
\end{center}

\begin{flushright}
\LARGE{Выполнил студент группы P3109 \\ 
Евграфов Артём Андреевич \\
Проверила: \\
Воронина Дарья Сергеевна}
\vspace{120pt}
\end{flushright}

\begin{center}
\Large{Санкт-Петербург 2025}
\end{center}

\newpage
\setcounter{page}{1}
\tableofcontents
\newpage
\section{Задание варианта 1986}
Для выполнения лабораторной работы №1 \underline{необходимо}:
\begin{enumerate}
    \item На основе предложенной предметной области (текста) составить её описание. Из полученного
    описания выделить сущности, их атрибуты и связи.
    \item Составить инфологическую модель.
    \item Составить даталогическую модель. При описании типов данных для атрибутов должны использоваться типы из СУБД PostgreSQL.
    \item  Реализовать даталогическую модель в PostgreSQL. При описании и реализации даталогической модели должны учитываться ограничения целостности, которые характерны для полученной предметной  области.
    \item Заполнить созданные таблицы тестовыми данными.
\end{enumerate}

\noindent Для создания объектов базы данных у каждого студента есть своя схема. Название схемы соответствует имени пользователя в базе studs (sXXXXXX). Команда для подключения к базе studs: \\

\noindent psql -h pg -d studs \\

\noindent Каждый студент должен использовать свою схему при работе над лабораторной работой №1 (а также
в рамках выполнения 2, 3 и 4 этапов курсовой работы). \\

\noindent \underline{Отчёт по лабораторной работе должен содержать:} 
\begin{enumerate}
    \item Текст задания.
    \item Описание предметной области.
    \item Список сущностей и их классификацию (стержневая, ассоциация, характеристика).
    \item Инфологическая модель (ER-диаграмма в расширенном виде - с атрибутами, ключами. . . ).
    \item Даталогическая модель (должна содержать типы атрибутов, вспомогательные таблицы для отображения связей "многие-ко-многим").
    \item Реализация даталогической модели на SQL.
    \item Выводы по работе.
\end{enumerate}
\section{Описание предметной области}
\subsection{Текст задания}
Еще несколько минут разгона - и "Леонов"\, выйдет на долгую дорогу домой. Флойд ощущал огромное облегчение. Подчиняясь законам небесной механики, корабль проследует через всю Солнечную систему, мимо бредущих по своим причудливым орбитам астероидов, мимо Марса, и ничто не остановит его на пути к Земле.
\subsection{Описание}
Флойд - человек, присутствует на корабле "Леонов". Каждому кораблю соответствует команда, во время выполнения рейса на корабле должна присутствовать единственная команда (в последующих рейсах команда может быть другая). Каждый человек входит в какую-то команду (возможно, не одну). Рейс состоит из объектов, каждый объект находится в какой-то системе. \\
Стержневые сущности: Route, Team, Person, SpaceObject. \\
Ассоциативные сущности: RoutesStep, PersonAndTeam, Flights. \\
Характеристические сущности: SpaceSystems.\\
\newpage
\section{Инфологическая модель}
\begin{figure}[H]
    \centering
\includegraphics[width=1\linewidth]{IM.png}
\end{figure}
\newpage
\section{Даталогическая модель}
\begin{figure}[H]
    \centering
\includegraphics[width=1\linewidth]{DM.png}
\end{figure}
\section{Реализация модели на SQL}
\begin{lstlisting}
DROP TABLE IF EXISTS SpaceSystem CASCADE;
DROP TABLE IF EXISTS SpaceObject CASCADE;
DROP TABLE IF EXISTS Ship CASCADE;
DROP TABLE IF EXISTS Route CASCADE;
DROP TABLE IF EXISTS RoutesStep CASCADE;
DROP TABLE IF EXISTS Team CASCADE;
DROP TABLE IF EXISTS Person CASCADE;
DROP TABLE IF EXISTS PersonAndTeam CASCADE;
DROP TABLE IF EXISTS Flights CASCADE;


CREATE TABLE SpaceSystem
(
    systemID int PRIMARY KEY,
    name     varchar(40) NOT NULL
);

CREATE TABLE SpaceObject
(
    objectID int PRIMARY KEY,
    systemID int         NOT NULL,
    FOREIGN KEY (systemID) REFERENCES SpaceSystem (systemID) ON DELETE CASCADE,
    type     varchar(40) NOT NULL
);

CREATE TABLE Ship
(
    shipID integer PRIMARY KEY,
    name   varchar(30) NOT NULL
);

CREATE TABLE Route
(
    routeID     integer PRIMARY KEY,
    description varchar(100) NOT NULL
);

CREATE TABLE RoutesStep
(
    stepID   integer PRIMARY KEY,
    routeID  integer NOT NULL,
    FOREIGN KEY (routeID) REFERENCES Route (routeID) ON DELETE CASCADE,
    objectID integer NOT NULL,
    FOREIGN KEY (objectID) REFERENCES SpaceObject (objectID) ON DELETE CASCADE ,
    step     integer NOT NULL
);

CREATE TABLE Team
(
    teamID integer PRIMARY KEY
);

CREATE TABLE Person
(
    personID integer PRIMARY KEY,
    name     varchar(40) NOT NULL,
    sex      varchar(6)  NOT NULL,
    age      int         NOT NULL,
    shipRole varchar(40) NOT NULL
);

CREATE TABLE PersonAndTeam
(
    PandTID  integer PRIMARY KEY,
    personID integer NOT NULL,
    FOREIGN KEY (personID) REFERENCES Person (personID) ON DELETE CASCADE,
    teamID   integer NOT NULL,
    FOREIGN KEY (teamID) REFERENCES Team (teamID) ON DELETE CASCADE
);

CREATE TABLE Flights
(
    flightID  integer PRIMARY KEY,
    teamID    integer   NOT NULL,
    FOREIGN KEY (teamID) REFERENCES Team (teamID) ON DELETE CASCADE,
    shipID    integer   NOT NULL,
    FOREIGN KEY (shipID) REFERENCES Ship (shipID) ON DELETE CASCADE,
    routeID   integer   NOT NULL,
    FOREIGN KEY (routeID) REFERENCES Route (routeID) ON DELETE CASCADE,
    startTime timestamp NOT NULL,
    endTime   timestamp NOT NULL
);

INSERT INTO SpaceSystem (systemID, name)
VALUES (1, 'Solar System'),
       (2, 'Alpha Centauri'),
       (3, 'Andromeda');

INSERT INTO SpaceObject (objectID, systemID, type)
VALUES (1, 1, 'Planet'),
       (2, 1, 'Moon'),
       (3, 2, 'Exoplanet'),
       (4, 3, 'Black Hole');

INSERT INTO Ship (shipID, name)
VALUES (1, 'Apollo 11'),
       (2, 'Enterprise'),
       (3, 'Millennium Falcon'),
       (4, 'Voyager');

INSERT INTO Route (routeID, description)
VALUES (1, 'Basic route'),
       (2, 'Mars Exploration'),
       (3, 'Deep Space Mission');

INSERT INTO RoutesStep (stepID, routeID, objectID, step)
VALUES (1, 1, 1, 1),
       (2, 1, 2, 2),
       (3, 2, 3, 1),
       (4, 3, 4, 1),
       (5, 1, 3, 3);

INSERT INTO Team (teamID)
VALUES (1),
       (2),
       (3);

INSERT INTO Person (personID, name, sex, age, shipRole)
VALUES (1, 'Neil Armstrong', 'Male', 39, 'Commander'),
       (2, 'Buzz Aldrin', 'Male', 38, 'Pilot'),
       (3, 'Yuri Gagarin', 'Male', 34, 'Engineer'),
       (4, 'Sally Ride', 'Female', 32, 'Scientist');

INSERT INTO PersonAndTeam (PandTID, personID, teamID)
VALUES (1, 1, 1),
       (2, 2, 1),
       (3, 3, 2),
       (4, 4, 3),
       (5, 4, 2);


INSERT INTO Flights (flightID, teamID, shipID, routeID, startTime, endTime)
VALUES (1, 1, 1, 1, '1969-07-16 12:43:12', '1969-07-24 12:53:14'),
       (2, 2, 2, 2, '2030-05-10 15:23:15', '2030-06-15 11:43:12'),
       (3, 3, 3, 3, '2100-01-01 22:43:12', '2102-12-31 12:43:52'),
       (4, 3, 2, 1, '2100-01-01 12:49:14', '2102-12-31 02:43:12');
\end{lstlisting}
\section{Вывод}
Во время выполнения данной лабораторной работы я научился составлять инфологическую и даталогическую модели сущностей, научился реализовывать даталогические модели произвольной предметной области с помощью SQL.
\end{document}