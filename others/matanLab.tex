\documentclass[14pt,final,oneside]{article}% класс документа, характеристики
%>>>>> Разметка документа
\usepackage[a4paper, mag=1000, left=1cm, right=1cm, top=1cm, bottom=1cm, headsep=0.7cm, footskip=1cm]{geometry} % По ГОСТу: left>=3cm, right=1cm, top=2cm, bottom=2cm,
\linespread{1} % межстройчный интервал по ГОСТу := 1.5
%<<<<< Разметка документа
\usepackage[utf8]{inputenc}
\usepackage[T2A]{fontenc}
\usepackage[english, russian]{babel}
\usepackage{amsmath, amsfonts, amssymb, amssymb, amsthm, mathtools}
\usepackage{geometry}
\pagenumbering{gobble} % откл нумерацию
\usepackage{colortbl} % таблички
\usepackage{listings} % листинг
\usepackage{dcolumn} % выравнивание чисел
\usepackage[normalem]{ulem} % для подчёркиваний uline
\ULdepth = 0.16em % расстояние от линии до текста выше/ниже
\lstset{basicstyle=\ttfamily\normalsize}  
\usepackage{graphicx}
\usepackage{xcolor}
\usepackage{float}
\lstset{
    language=Java,
    numbers=left,                    % Нумерация строк слева
    stepnumber=1,                    % Каждая строка нумеруется
    numbersep=5pt,                   % Отступ от кода до номеров строк
    showspaces=false,                % Не показывать пробелы
    showstringspaces=false,          % Не показывать пробелы в строках
    tabsize=4,                       % Размер табуляции
    breaklines=true,                 % Перенос строк
    breakatwhitespace=false,         % Переносить строки по пробелам
    basicstyle=\ttfamily,            % Шрифт текста
    keywordstyle=\color{blue},       % Цвет ключевых слов
    commentstyle=\color{green},      % Цвет комментариев
    stringstyle=\color{red},         % Цвет строковых литералов
}
\usepackage{titlesec}
\titleformat{\section}
{\LARGE\bfseries}
{\thesection}{15pt}{} 
\usepackage{fancyhdr}
\theoremstyle{theorem}
\newtheorem{theorem}{Теорема}
\newtheorem{lemma}{Лемма}
\renewcommand{\thesection}{\arabic{section}.}
\renewcommand{\thesubsection}{\arabic{section}.\arabic{subsection}.}
\usepackage{tocloft}
\renewcommand{\cftsecleader}{\cftdotfill{\cftdotsep}}
\renewcommand{\cftsecfont}{\Large\bfseries} 
\renewcommand{\cfttoctitlefont}{\LARGE\bfseries}
\newcommand{\lr}[1]{\left( {#1} \right)}
\usepackage{hyperref} % гиперссылки
\hypersetup{
    colorlinks=true,        % Включаем цветные ссылки
    linkcolor=blue!50!black, % Цвет внутренних ссылок (например, на разделы)
    urlcolor=blue!50!black,  % Цвет URL-ссылок
    citecolor=blue!50!black, % Цвет ссылок на библиографию
}

\begin{document}
\hspace{1cm}
\thispagestyle{empty}
\begin{center}
\LARGE{Университет ИТМО} 
\vspace{20pt}

\LARGE{Факультет программной инженерии и компьютерной техники \\
Образовательная программа системное и прикладное программное обеспечение}
\vspace{160pt}

\LARGE{Лабораторная работа  \textnumero 1 \\
По дисциплине "Математический анализ" \\ 
Вариант 26}
\vspace{120pt}
\end{center}

\begin{flushright}
\LARGE{Выполнил студент группы P3109 \\ 
Евграфов Артём Андреевич (ISU: 465826)}
\vspace{120pt}
\end{flushright}

\begin{center}
\Large{Санкт-Петербург 2024}
\end{center}

\newpage
\setcounter{page}{1}
\Large
\noindent Изначально в варианте 26 мне была дана функция $f(x)=\ln{\frac{1+x}{1-x}} + \frac{1}{x}$, однако она не имеет действительных корней. Докажем это. \\
$D(f(x))=(-1; 1)$, заметим, что $f(x)=-f(-x)\text{, так как } f(-x)=\ln{\frac{1-x}{1+x}}-\frac{1}{x}=\ln{((\frac{1+x}{1-x})^{-1})} - \frac{1}{x} = -(\ln{\frac{1+x}{1-x}} + \frac{1}{x}) = -f(x)$, те $f(x)$ - нечётная ф-ция. Рассмотрим $x \in (0; 1)$. \\
Найдем производную функции:
\[
f(x) = \ln\left(\frac{1+x}{1-x}\right) + \frac{1}{x}
\]

\noindent Воспользуемся правилом дифференцирования. Производная суммы равна сумме производных:

\[
f'(x) = \frac{d}{dx}\left[\ln\left(\frac{1+x}{1-x}\right)\right] + \frac{d}{dx}\left[\frac{1}{x}\right]
\]

\noindent Для первой части:
\[
\frac{d}{dx}\left[\ln\left(\frac{1+x}{1-x}\right)\right] = \frac{1}{\frac{1+x}{1-x}} \cdot \frac{d}{dx}\left[\frac{1+x}{1-x}\right]
\]
\[
\frac{d}{dx}\left[\frac{1+x}{1-x}\right] = \frac{(1-x) - (1+x)(-1)}{(1-x)^2} = \frac{1-x+1+x}{(1-x)^2} = \frac{2}{(1-x)^2}
\]
Подставляем:
\[
\frac{1}{\frac{1+x}{1-x}} \cdot \frac{2}{(1-x)^2} = \frac{(1-x)}{(1+x)} \cdot \frac{2}{(1-x)^2} = \frac{2}{(1+x)(1-x)}
\]

\noindent Для второй части:
\[
\frac{d}{dx}\left[\frac{1}{x}\right] = -\frac{1}{x^2}
\]

\noindent Итак, производная:
\[
f'(x) = \frac{2}{(1+x)(1-x)} - \frac{1}{x^2} = \frac{3x^2-1}{x^2(1-x^2)}.
\]

\noindent Заметим, что при заданных х знаменатель производной строго больше 0, а чиститель равен 0 при $x = \frac{\sqrt{3}}{3}$, причём при $x < \frac{\sqrt{3}}{3}: \: f'(x) < 0 \text{, а при } x > \frac{\sqrt{3}}{3}: \: f'(x) > 0$, значит $x = \frac{\sqrt{3}}{3}$ - точка минимума функции при х из интервала $(0; 1)$, значит $\forall x \in D(f(x)) \: |{f(x)|} \geq f(\frac{\sqrt{3}}{3}) = \ln{(2+\sqrt{3})} + \sqrt{3} > 0$. Это доказывает, что действительных корней нет. \\
Вследствие того, что у предложенной для 26 варианта функции нет действительных корней, в дальнейшем я буду рассматривать функцию из 9 варианта. 
\newpage
Рассмотрим $f(x) = x^3 - 3x - 2e^{-x}$. $D(f(x)) = \mathbb{R}$
\begin{figure}[H]
    \centering
    \includegraphics[width=0.45\linewidth]{image.png}
    \caption{График $y = f(x)$}
\end{figure}
\noindent График функции пересекает ось Оx между $x=1$ и $x=2$. \\
Проверим удовлетворение условий теоремы Больцано-Коши: \\
$f(1) = 1 - 3 - 2e^{-1} \approx -2.73576$; \\
$f(2) = 2^3 - 3\cdot2 - 2e^{-2} \approx 1.72933$; \\
$f(1)f(2)<0$, значит $\exists c: f(c) = 0$. \\
Программа на языке программирования Python, которая позволяет находить корень с точностью $10^{-17}$: 
\begin{lstlisting}
from math import *
from decimal import *

def f(x):
    x = Decimal(str(x))
    return x ** Decimal("3") - Decimal("3") * x - Decimal("2") * Decimal(str(pow((e), (-x))))

def bisect(l, r, eps):
    mid = (l + r) / 2
    while (abs(f(mid)) > eps):
        if (f(mid) > 0):
            r = mid
        else:
            l = mid
        mid = (l + r) / 2
    return (r + l) / 2


print(bisect(Decimal(1), Decimal(2),
             Decimal(str(10 ** (-17)))))
\end{lstlisting}
Программа выводит результат: 1.785461454619108521285619772; \\
Результат, выводимый WolframAlpha: 1.78546145461910852128561977296418336576. \\
Вывод: \\
В ходе лабораторной работы я научился приблизительно находить корни уравнения $f(x)=0$, используя метод бисекций.
\end{document}

\end{document}
