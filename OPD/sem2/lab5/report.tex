\documentclass[14pt,final,oneside]{article}% класс документа, характеристики
%>>>>> Разметка документа
\usepackage[a4paper, mag=1000, left=3cm, right=1.5cm, top=2cm, bottom=2cm, headsep=0.7cm, footskip=1cm]{geometry} % По ГОСТу: left>=3cm, right=1cm, top=2cm, bottom=2cm,
\linespread{1} % межстройчный интервал по ГОСТу := 1.5
%<<<<< Разметка документа
\usepackage[utf8]{inputenc}
\usepackage[T2A]{fontenc}
\usepackage[english, russian]{babel}
\usepackage{amsmath, amsfonts, amssymb, amssymb, amsthm, mathtools}
\usepackage{geometry}
\usepackage{colortbl} % таблички
\usepackage{listings} % листинг
\usepackage{dcolumn} % выравнивание чисел
\usepackage[normalem]{ulem} % для подчёркиваний uline
\ULdepth = 0.16em % расстояние от линии до текста выше/ниже
\lstset{basicstyle=\ttfamily\normalsize}  
\usepackage{graphicx}
\usepackage{xcolor}
\usepackage{float}
\lstset{
    language=Java,
    numbers=left,                    % Нумерация строк слева
    stepnumber=1,                    % Каждая строка нумеруется
    numbersep=5pt,                   % Отступ от кода до номеров строк
    showspaces=false,                % Не показывать пробелы
    showstringspaces=false,          % Не показывать пробелы в строках
    tabsize=4,                       % Размер табуляции
    frame=single
    breaklines=true,                 % Перенос строк
    breakatwhitespace=false,         % Переносить строки по пробелам
    basicstyle=\ttfamily,            % Шрифт текста
    keywordstyle=\color{blue},       % Цвет ключевых слов
    commentstyle=\color{green},      % Цвет комментариев
    stringstyle=\color{red},         % Цвет строковых литералов
}
\usepackage{titlesec}
\titleformat{\section}
{\LARGE\bfseries}
{\thesection}{15pt}{} 
\usepackage{fancyhdr}
\theoremstyle{theorem}
\newtheorem{theorem}{Теорема}
\newtheorem{lemma}{Лемма}
\renewcommand{\thesection}{\arabic{section}.}
\renewcommand{\thesubsection}{\arabic{section}.\arabic{subsection}.}
\usepackage{tocloft}
\renewcommand{\cftsecleader}{\cftdotfill{\cftdotsep}}
\renewcommand{\cftsecfont}{\Large\bfseries} 
\renewcommand{\cfttoctitlefont}{\LARGE\bfseries}
\newcommand{\lr}[1]{\left( {#1} \right)}
\usepackage{hyperref} % гиперссылки
\hypersetup{
    colorlinks=true,        % Включаем цветные ссылки
    linkcolor=blue!50!black, % Цвет внутренних ссылок (например, на разделы)
    urlcolor=blue!50!black,  % Цвет URL-ссылок
    citecolor=blue!50!black, % Цвет ссылок на библиографию
}

\begin{document}
\thispagestyle{empty}
\begin{center}
\LARGE{Университет ИТМО} 
\vspace{20pt}

\LARGE{Факультет программной инженерии и компьютерной техники \\
Образовательная программа системное и прикладное программное обеспечение}
\vspace{160pt}

\LARGE{Лабораторная работа  \textnumero 5 \\
По дисциплине "Основы профессиональной деятельности" \\ 
Вариант 9500}
\vspace{120pt}
\end{center}

\begin{flushright}
\LARGE{Выполнил студент группы P3109 \\ 
Евграфов Артём Андреевич \\
Проверила: \\
Ткешелашвили Нино Мерабиевна}
\vspace{120pt}
\end{flushright}

\begin{center}
\Large{Санкт-Петербург 2025}
\end{center}

\newpage
\setcounter{page}{1}
\tableofcontents
\newpage
\section{Задание варианта 9500}
\begin{figure}[H]
    \centering
    \includegraphics[width=1\linewidth]{P1O.png}
\end{figure}
\section{Описание программы}
\begin{longtable}{|>{\centering\arraybackslash}p{1cm}|>{\centering\arraybackslash}p{2cm}|>{\centering\arraybackslash}p{3cm}|>{\arraybackslash}p{9cm}|}
\hline
Адрес & Содержимое & Мнемоника & Комментарии \\
\hline
\endfirsthead

\hline
Адрес & Содержимое & Мнемоника & Комментарии \\
\hline
\endhead

\hline
\endfoot

\hline
\endlastfoot
28B & 05CD & ADR & Ячейка для инкрементирования адреса результата  \\\hline
28C & 0200 & CLA & 0000 -> AC \\\hline
\rowcolor{yellow}
28D & 1205 & IN 5 & Ввод содержимого SR ВУ2 в 6-й бит AC \\\hline
\rowcolor{yellow}
28E & 2F40 & AND \#40 & Проверяем условие "6-й бит AC = 1"  \\\hline
\rowcolor{yellow}
28F & F0FD & BZS (IP - 3) & Переход на 0x28D, если 6-й бит = 0 (Z == 1) \\\hline
\rowcolor{green}
290 & 1204 & IN 4 & Ввод содержимого DR ВУ2 в младший байт AC \\\hline
\rowcolor{green}
291 & 7F0A & CMP \#0A & Сравнение AC со стоп-символом 0A \\\hline
\rowcolor{green}
292 & F00B & BZS (IP + 11) & Переход на 0x29E, если Z == 1 \\\hline
293 & 0680 & SWAB & Свап байтов AC \\\hline
294 & E8F6 & ST (IP - 10) & Сохранение первого символа по адресу из ячейки 0x28B  \\\hline
\rowcolor{yellow}
295 & 1205 & IN 5 & Ввод содержимого SR ВУ2 в 6-й бит AC \\\hline
\rowcolor{yellow}
296 & 2F40 & AND \#40 & Проверяем условие "6-й бит AC = 1" (старший байт затирается из-за расширения знака)  \\\hline
\rowcolor{yellow}
297 & F0FD & BZS (IP - 3) & Переход на 0x294, если 6-й бит = 0 (Z == 1) \\\hline
\rowcolor{green}
298 & 1204 & IN 4 & Ввод содержимого DR ВУ2 в младший байт AC \\\hline
\rowcolor{green}
299 & 7F0A & CMP \#0A & Сравнение AC со стоп-символом 0A \\\hline
\rowcolor{green}
29A & F005 & BZS (IP + 5) & Переход на 0x29E, если Z == 1 \\\hline
29B & 48EF & ADD (IP - 17) & Теперь в AC хранится СИМВ1 СИМВ2 \\\hline
29C & EAEE & ST (IP - 18)+ & Сохранение второго символа по адресу из ячейки 0x28B и увеличение значения ячейки на 1 \\\hline
29D & CEED & JUMP (IP - 18) & Переход на адрес 0x28D (если все окей, продолжаем считывать)  \\\hline
29E & 0680 & SWAB & Свап байтов АС (так как 0A должен быть слева) \\\hline
29F & CE01 & JUMP (IP + 1) & Переходим на 0x2A0  \\\hline
2A0 & 48EA & ADD (IP - 22) & Теперь в AC хранится СИМВ1 СИМВ2=0A \\\hline
2A1 & E8E9 & ST (IP - 23) & Сохранение результата с символом 0A  \\\hline 
2A2 & 0100 & HLT & остановка программы \\\hline 
 \rowcolor{red}
5CD & 0000 & RES & Ячейка для сохранения символа слова (дальше инкрементируется) 
\end{longtable}
\section{ОП и ОДЗ исходных данных и результата}
\subsection{Область представления}
RES - 16-разрядная ячейка для сохранения 2-х символов. Старший байт - код первого символа, младший байт - код второго символа. \\
ADR - 11-разрядное беззнаковое число. Ячейка для хранения адреса начала символов кода.
\begin{figure}[H]
    \centering
    \includegraphics[width=1\linewidth]{KOI.png}
\end{figure}
\subsection{Область определения}
8-ричный код символа $\in [\text{0x20}; \text{0xFF}\textbackslash\{0x7\text{F}\}]$. \\
Максимально возможное количество символов для ввода = 1126.\\
$(2047(0x7FF_{16}) - 1484(0x5CD_{16}))\cdot2 =1126$
\section{Трассировка программы} 
Слово для трассировки: СМЕСЬ. \\
Слово в кодировке КОИ-8: F3 ED E5 F3 F8. \\
Слово в кодировке UTF-8 (BE):  D0A1 D09C D095 D0A1 D0AC\\
Слово в кодировке UTF-16 (BE): 0421 041C 0415 0421 042C \\
Таблица трассировки первых двух символов слова СМЕСЬ:
\begin{longtable}{|*{12}{>{\ttfamily}r|}}
\hline
\textbf{Адр} & \textbf{Знчн} & \textbf{IP} & \textbf{CR} & \textbf{AR} & \textbf{DR} & \textbf{SP} & \textbf{BR} & \textbf{AC} & \textbf{NZVC} & \textbf{Адр} & \textbf{Знчн} \\\hline
28B & 05CD & 28B & 0000 & 000 & 0000 & 000 & 0000 & 0000 & 0100 &  & \\\hline
28B & 05CD & 28C & 05CD & 28B & 05CD & 000 & 028B & 0000 & 0100 &  & \\\hline
28C & 0200 & 28D & 0200 & 28C & 0200 & 000 & 028C & 0000 & 0100 &  & \\\hline
28D & 1205 & 28E & 1205 & 28D & 1205 & 000 & 028D & 0040 & 0100 &  & \\\hline
28E & 2F40 & 28F & 2F40 & 28E & 0040 & 000 & 0040 & 0040 & 0000 &  & \\\hline
28F & F0FD & 290 & F0FD & 28F & F0FD & 000 & 028F & 0040 & 0000  &  & \\\hline
290 & 1204 & 291 & 1204 & 290 & 1204 & 000 & 0290 & 00F3 & 0000 &  & \\\hline
291 & 7F0A & 292 & 7F0A & 291 & 000A & 000 & 000A & 00F3 & 0001 &  & \\\hline
292 & F00B & 293 & F00B & 292 & F00B & 000 & 0292 & 00F3 & 0001 &  & \\\hline
293 & 0680 & 294 & 0680 & 293 & 0680 & 000 & 0293 & F300 & 1001 &  & \\\hline
294 & E8F6 & 295 & E8F6 & 5CD & F300 & 000 & FFF6 & F300 & 1001 & 5CD & F300 \\\hline
295 & 1205 & 296 & 1205 & 295 & 1205 & 000 & 0295 & F300 & 1001 &  & \\\hline
296 & 2F40 & 297 & 2F40 & 296 & 0040 & 000 & 0040 & 0000 & 0101 &  & \\\hline
297 & F0FD & 295 & F0FD & 297 & F0FD & 000 & FFFD & 0000 & 0101 &  & \\\hline
295 & 1205 & 296 & 1205 & 295 & 1205 & 000 & 0295 & 0040 & 0101 &  & \\\hline
296 & 2F40 & 297 & 2F40 & 296 & 0040 & 000 & 0040 & 0040 & 0001 &  & \\\hline
297 & F0FD & 298 & F0FD & 297 & F0FD & 000 & 0297 & 0040 & 0001 &  & \\\hline
298 & 1204 & 299 & 1204 & 298 & 1204 & 000 & 0298 & 00ED & 0001 &  & \\\hline
299 & 7F0A & 29A & 7F0A & 299 & 000A & 000 & 000A & 00ED & 0001 &  & \\\hline
29A & F005 & 29B & F00B & 29A & F00B & 000 & 029A & 00ED & 0001 &  & \\\hline
29B & 48EF & 29C & 48EF & 5CD & F300 & 000 & FFEF & F3ED & 1000 &  & \\\hline
29C & EAEE & 29D & EAEE & 5CD & F3ED & 000 & FFEE & F3ED & 1000 & 28B & 05CE  \\\hline
& & & & & & & & & & 5CD & F3ED \\\hline
29D & CEEF & 28D & CEEF & 29D & 028D & 000 & FFEF & F3ED & 1000 & & \\\hline
28D & 1205 & 28E & 1205 & 28D & 1205 & 000 & 028D & F340 & 1000 &  & \\\hline
28E & 2F40 & 28F & 2F40 & 28E & 0040 & 000 & 0040 & 0040 & 0000 &  & \\\hline
28F & F0FD & 290 & F0FD & 28F & F0FD & 000 & 028F & 0040 & 0000 &  & \\\hline
290 & 1204 & 291 & 1204 & 290 & 1204 & 000 & 0290 & 000A & 0000 &  & \\\hline
291 & 7F0A & 292 & 7F0A & 291 & 000A & 000 & 000A & 000A & 0101 &  & \\\hline
292 & F00B & 29E & F00B & 292 & F00B & 000 & 000B & 000A & 0101 &  & \\\hline
29E & 0680 & 29F & 0680 & 29E & 0680 & 000 & 029E & 0A00 & 0001 &  & \\\hline
29F & CE01 & 2A1 & CE01 & 29F & 02A1 & 000 & 0001 & 0A00 & 0001 &  & \\\hline
2A1 & E8E9 & 2A2 & E8E9 & 5CE & 0A00 & 000 & FFE9 & 0A00 & 0001 & 5CE & 0A00 \\\hline
2A2 & 0100 & 2A3 & 0100 & 2A2 & 0100 & 000 & 02A2 & 0A00 & 0001 &  & \\\hline
\end{longtable}

\section{Дополнительное задание}
С ВУ-8 (клавиатура) вводится строка, enter - завершение ввода. После окончания ввода, на ВУ-5 (принтер) вывести коды символов в 16-ричной системе счисления через пробел. Кодировка любая.
\begin{lstlisting}
ORG 0x000
ADR: WORD $RES
COUNTER: WORD 0x0000

BEGIN: CLA

READ: IN 0x19
      AND #0x40
      BZS READ
      IN 0x18
      CMP #0x0A
      BZS SAVE
      ST (ADR)+
      LD COUNTER
      INC
      ST COUNTER
      JUMP READ

SAVE: CLA
      LD COUNTER
      CMP #0x00 
      BZS EXIT
      JUMP START_WRITING

START_WRITING: CLA
      IN 0xD
      AND #0x40
      BZS START_WRITING
      LD ADR
      SUB COUNTER
      ST ADR
      JUMP WRITE

WRITE: CLA
      LD (ADR)
      ASR
      ASR
      ASR
      ASR
      PUSH
      CALL FUNC
      POP
      OUT 0xC
      LD (ADR)+
      AND #0x0F
      PUSH
      CALL FUNC
      POP
      OUT 0xC
      LD #0x9A
      OUT 0xC
      LOOP COUNTER
      JUMP WRITE
      HLT        

FUNC: LD &1
      CMP #0x00
      BZS SET_0
      CMP #0x01
      BZS SET_1
      CMP #0x02
      BZS SET_2
      CMP #0x03
      BZS SET_3
      CMP #0x04
      BZS SET_4
      CMP #0x05
      BZS SET_5
      CMP #0x06
      BZS SET_6
      CMP #0x07
      BZS SET_7
      CMP #0x08
      BZS SET_8
      CMP #0x09
      BZS SET_9
      CMP #0x0A
      BZS SET_A
      CMP #0x0B
      BZS SET_B
      CMP #0x0C
      BZS SET_C
      CMP #0x0D
      BZS SET_D
      CMP #0x0E
      BZS SET_E
      CMP #0x0F
      BZS SET_F  
        
SET_0:  LD #0x30
        ST &1
        RET
        
SET_1:  LD #0x31
        ST &1
        RET
        
SET_2:  LD #0x32
        ST &1
        RET
        
SET_3:  LD #0x33
        ST &1
        RET
        
SET_4:  LD #0x34
        ST &1
        RET
        
SET_5:  LD #0x35
        ST &1
        RET
        
SET_6:  LD #0x36
        ST &1
        RET
        
SET_7:  LD #0x37
        ST &1
        RET
        
SET_8:  LD #0x38
        ST &1
        RET
        
SET_9:  LD #0x39
        ST &1
        RET
        
SET_A:  LD #0x41
        ST &1
        RET
        
SET_B:  LD #0x42
        ST &1
        RET
        
SET_C:  LD #0x43
        ST &1
        RET
        
SET_D:  LD #0x44
        ST &1
        RET
        
SET_E:  LD #0x45
        ST &1
        RET
        
SET_F:  LD #0x46
        ST &1
        RET
          
ORG 0x2CD
RES: WORD 0x0000
\end{lstlisting}
\section{Вывод}
В ходе выполнения данной лабораторной работы я научился работать с ВУ-2, освоил команды ввода-вывода, а также познакомился с синтаксисом ассемблера БЭВМ-NG.
\end{document}